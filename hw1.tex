% TODO: check spelling!!!
\documentclass{article}
\usepackage{enumitem} % Change enumerate
\setlist[enumerate,2]{label=(\alph*)} % Change enumerate
\usepackage{amsmath}
\title{MATH 545: Homework 1}
\author{Yingli, Laurette, Fernando}
\date{September 9th, 2023}
\begin{document}
\maketitle
\begin{enumerate}
\item Solve the following ODEs.
\begin{enumerate}
	\item $(y')^2 -4y +4 = 0$

		Solution:

		We proceed by separation of variables.

		\begin{align*}
			(y')^2 -4y +4 = 0 \\
			\iff (y')^2 = 4y -4
		\end{align*}
		This means that we have two options for $y'$, namely:
		\[y' = \pm 2\sqrt{y-1}.\]
		Then:
		\begin{align*}
			y' = \pm 2\sqrt{y-1} \\
			\iff \frac{dy}{dx} = \pm 2\sqrt{y-1} \\
			\iff \frac{dy}{\pm 2\sqrt{y-1}} = dx
		\end{align*}
		Integrating we get:
		$\pm \sqrt{y-1} = x+c$,

		then
		\begin{equation} \label{sol}
			y = (x+c)^2 + 1.
		\end{equation}
		Now we test the function we found. First we notice that
		\begin{equation} \label{sol_deriv}
			y' = 2(x+c),
		\end{equation}
		then we substitute (\ref{sol}) and (\ref{sol_deriv}) in the original equation to obtain
		\begin{align*}
			(2(x+c))^2 -4((x+c)^2 + 1) +4 =0 \\
			\iff 4(x+c)^2 -4(x+c)^2 -4 + 4 =0,
		\end{align*}
		so the function we found is in fact a solution.

	\item $(y\;\text{sin} x + xy\;\text{cos}x)dx + (x\;\text{sin}x + 1)dy = 0$

		Solution:
		
		We start by showing that this equation is exact.

		In this case we have $M(x,y) = y\text{sin}x + xy\text{cos}x$ and $N(x,y) = x\text{sin}x + 1$.

		Then $\frac{\partial M(x,y)}{\partial y} = \text{sin}x + x\text{cos}x$ and $\frac{\partial N(x,y)}{\partial x} = \text{sin}x + x\text{cos}x$

		i.e. $\frac{\partial M(x,y)}{\partial y} = \frac{\partial N(x,y)}{\partial x}$.

		Hence, the equation is exact.

		Now we solve $\frac{\partial f(x,y)}{\partial y} = x\text{sin}x + 1$. Integrating with respect to $y$ we get
		\[f(x,y) = y(x\text{sin}x +1) + h(x)\]
		In order to determine $h(x)$ we differentiate with
		respect to $x$ to obtain
		\[\frac{\partial f(x,y)}{\partial x} = y(\text{sin}x+x\text{cos}x) + h'(x).\]
		This must equal $M(x,y)$, so:
		\[y(\text{sin}x+x\text{cos}x) + h'(x) = y\text{sin}x + xy\text{cos}x,\]
		this implies that $h'(x)=0$, which means that $h(x) = c_1$.
		So the solution is implicitely given by:
		\begin{align*}
			f(x,y) = y(x\text{sin}x +1) + c_1 = c_2 \\
			\iff y(x\text{sin}x +1) = c \\
		\end{align*}
		In order to verify that this function solves the equation we calculate the differential.

		\[\frac{\partial f}{\partial x} = y\text{sin}x + xy\text{cos}x\]
		\[\frac{\partial f}{\partial y} = x\text{sin}x + 1\]
		So when we calculate $df$ we get:
		\[ df = (y\text{sin}x + xy\text{cos}x)dx + (x\text{sin}x + 1)dy =0\]
		Hence, this solves the equation.
		
	\item $y' + (2/x)y = x$; initial condition $y(1)=0$

		Solution:

		We begin by solving the homogeneous equation

		\[ \frac{dy}{dx} + \frac{2y}{x} = 0\]

		by separation of variables

		\[\frac{dy}{y} = -2\frac{dx}{x} \implies ln(y) = -2ln(x) + c_1 \implies y = \frac{c_2}{x^2}.\]

		% TODO: FIX THIS!!!!
		% This can be simplified !!
		Now we solve the equation:

		\begin{equation} \label{sol_homo}
			y(x) = q(x)u(x)
		\end{equation}

		where $q(x) = \frac{1}{x^2}$. For this we substitute (\ref{sol_homo}) into the original equation.
		\begin{align}
			y' + (2/x)y = x \nonumber \\
			\iff q'(x)u(x) +q(x)u'(x) + (2/x)q(x)u(x) = x \nonumber \\
			\iff \frac{-2}{x^3}	u(x) +\frac{1}{x^2}u'(x) + (2/x)\frac{1}{x^2}u(x) = x \nonumber \\
			\iff \frac{1}{x^2}u'(x) = x \nonumber \\
			\iff u'(x) = x^3 \nonumber \\
			\implies u(x) = \frac{x^4}{4} + c \label{ux}.
		\end{align}

		Since we know $q(x)$ we can sustitute (\ref{ux}) into (\ref{sol_homo}) to get:
		\[y(x) = \frac{1}{x^2}\left(\frac{x^4}{4}+c\right) = \frac{x^2}{4} + \frac{c}{x^2}.\]

		The initial condition is $y(1)=0$ so $y(1)=\frac{1}{4} + c= 0 \implies c=-\frac{1}{4}$,
		then
		\[y(x) = \frac{1}{x^2}\left(\frac{x^4}{4}+c\right) = \frac{x^2}{4} - \frac{1}{4x^2}.\]

		Now we check that the function found is a solution.

		First, the initial condition:
		\[y(1)= \frac{1^2}{4} - \frac{1}{4\cdot 1^2} =0.\]

		Secondly, we check that it satisfies the equation.
		\[y'+(2/x)y = \left(\frac{x}{2} + \frac{1}{2x^3}\right) +(2/x)\left(\frac{x^2}{4} - \frac{1}{4x^2}\right) = x.\]
		Hence the function solves the equation.


\end{enumerate}
\item Strauss section 1.2
	\begin{enumerate}
		\item Exercise 1. Solve the first-order equation
			$2u_t +3u_ x =0$ with the auxiliary condition $u=\text{sin}x$ when $t=0.$

			Solution:

			We solved this type of equation in class.
			the solution is 

			$u(t,x)=f(2x-3t)$

			Where $f$ is any sufficiently nice function.

			Using the initial condition we get:
			\[u(0,x)=f(2x)=\text{sin}x,\]
			so $f(x)=\text{sin}\left(\frac{x}{2}\right)$.
			Finally we have:
			\[u(t,x)=\text{sin}\left(x-\frac{3}{2}t\right)\]
			We check by differentiating.

			\[u_x=\text{cos}\left(x-\frac{3}{2}t\right)\]
			\[u_t=-\frac{3}{2}\text{cos}\left(x-\frac{3}{2}t\right)\]

			so
			\[ 2u_t +3u_x=-3\text{cos}\left(x-\frac{3}{2}t\right) + 3\text{cos}\left(x-\frac{3}{2}t\right)=0\]

			As for the initial condition:
			\[u(0,x)=\text{sin}\left(x-\frac{3}{2}\cdot0\right)=\text{sin}x.\]

			Hence $u(t,x)$ is in fact a solution to the PDE.

		\item Exercise 2. Solve the equation $3u_y + u_{xy} =0$

			Solution:
			% TODO: Maybe we should frame this in terms of antiderivaties!!!

			Let $v=u_y$, then the equation becomes:
			% TODO this align looks ugly, make it nicer if time allows
			\begin{align*}
				 3v+v_x=0 \\
				 \iff 3v+\frac{\partial v}{\partial x}=0 \\
				 \iff \frac{\partial v}{\partial x}=-3v \\
				 \iff \frac{\partial v}{v}=-3\partial x \\
				 \iff ln(v) = -3x + f(y) \\
				 \iff v = g(y)e^{-3x}
			\end{align*}
			Since $v=u_y$ we have:
			\[u(x,y)=\int vdy = e^{-3x}\int g(y)dy + f(x).\]

		\item Exercise 6. Solve the equation $\sqrt{1-x^2}u_x+u_y=0$ with the condition $u(0, y)=y.$
		
            Solution: The charecteristic curve satisfies the ODE: 
            \[ \frac{dy}{dx}=\frac{1}{\sqrt{1-x^2}}\]
			To solve the solution, we separate variables: 		
			\[ dy=\frac{1}{\sqrt{1-x^2}}dx.\]
			By intergral, \[y= arcsinx + c .\]
			Since $u(x,y)$ is a constant on each such curve, so
			\[ u(x,y)=f(c)=f(y - arcsinx).\]
			Since the initial condition is $u(0,y)=y$, then $u(0,y)=f(y)=y.$

			Finally we have:
			\[u(t,x)=y - arcsinx.\]
			We check our solution by differentiating:
			\[u_x=-\frac{1}{\sqrt{1-x^2}}\]
			\[u_y=1\]
			so \[\sqrt{1-x^2}u_x+u_y=-\sqrt{1-x^2} \cdot \frac{1}{\sqrt{1-x^2}}+1=0.\]

			Hence $u(t,x)$ is in fact a solution to the PDE.


		\item Exercise 8. Solve $au_x+bu_y+cu=0.$
		Solution:
		By using coordinate method: let $x'=ax+by$, $y'=bx-ay$.

		% TODO: Change the variable to something else, like \tilda to avoid confusion with derivative
		Replace all $x$ and $y$ derivatives by $x'$ and $y'$ derivatives.

		By the chain rule, \[u_x=\frac{\partial u}{\partial x}=\frac{\partial u}{\partial x'}\cdot{\partial x'}{\partial x}+\frac{\partial u}{\partial y'}\cdot{\partial y'}{\partial x}=au_{x'}+bu_{y'},\]
		\[u_y=\frac{\partial u}{\partial y}=\frac{\partial u}{\partial x'}\cdot{\partial x'}{\partial y}+\frac{\partial u}{\partial y'}\cdot{\partial y'}{\partial y}=bu_{x'}-au_{y'}.\]

		Hence,
		% TODO: Improve this!!!
		\begin{align*}
			au_x+bu_y+cu=\\
			a\left(au_{x'}+bu_{y'} \right)+b\left(bu_{x'}-au_{y'}\right)=\left(a^2+b^2\right)u_{x'}+cu=0.
		\end{align*}
		We seperate variables, \[\frac{\partial u}{u}=-\frac{c}{a^2+b^2}\partial x'.\]
		Hence, \[\ln u = -\frac{c}{a^2+b^2}x' +g(y'),\] so that \[ u = f(y')e^{-\frac{c}{a^2+b^2}x'}.\]
		Thus the solution is  \[ u = f(bx-ay)e^{-\frac{c}{a^2+b^2}(ax+by)}.\]

		We check our solution by differentiating:
		\[u_x= b f'(bx-ay)e^{-\frac{c}{a^2+b^2}(ax+by)} -\frac{ac}{a^2+b^2} \cdot e^{-\frac{c}{a^2+b^2}(ax+by)}f(bx-ay),\]
		\[u_y= -a f'(bx-ay)e^{-\frac{c}{a^2+b^2}(ax+by)} -\frac{bc}{a^2+b^2} \cdot e^{-\frac{c}{a^2+b^2}(ax+by)}f(bx-ay),\]
		% TODO: Improve this!!
		So
		% TODO: ask Yingli if we should keep the e^f instead of just f
		\begin{align*}
			au_x+bu_y+cu= \\
			a\cdot b f'(bx-ay)e^{-\frac{c}{a^2+b^2}(ax+by)} \\
			-\frac{ac}{a^2+b^2} \cdot e^{-\frac{c}{a^2+b^2}(ax+by)}f(bx-ay) \\
			-b a f'(bx-ay)e^{-\frac{c}{a^2+b^2}(ax+by)} \\
			-\frac{bc}{a^2+b^2} \cdot e^{-\frac{c}{a^2+b^2}(ax+by)}f(bx-ay) \\
			= 0
		\end{align*}
	\end{enumerate}
\item Strauss section 1.3
	\begin{enumerate}
		\item Exercise 4.
		\item Exercise 8. For the hydrigen atom, if $ \int \left | u^2 \right | d \vec{x } =1$, at $t=0$, show thath the same is true at all later times.
		
		Solution: If we want to show that $ \int \left | u^2 \right | d \vec{x } =1$ is true at all times, we should show that $\frac{d}{dt}\int \left | u^2 \right | d \vec{x } =0$.
                  
		Since
		\begin{align*}
			 \frac{d}{dt}\int \left | u^2 \right | d \vec{x } =&
			 \int \frac{d}{dt} \left | u^2 \right | d \vec{x }=
			 \int \frac{d}{dt} \left(uu^* \right)d \vec{x }=
			 \int\left(  u_{t}u^*+uu_{t}^*\right) d \vec{x }
		\end{align*}

		According to Schr$\ddot{o}$dinger equation: 
		\[-\mathrm{i}hu_{t}=\frac{h^2}{2m}\triangle u+ \frac{e^2}{r}u,\]
		we will obtain that
		\[u_{t}=\frac{\mathrm{i}h}{2m}\triangle u+ \frac{ie^2}{hr}u,\]
		\[u_{t}^*=-\frac{\mathrm{i}h}{2m}\triangle u^*-\frac{ie^2}{hr}u^*,\]

		Therefore,

			\[u_{t}u^*+uu_{t}^*=(\frac{\mathrm{i}h}{2m}\triangle u+\frac{ie^2}{hr}u)u^*-u(\frac{\mathrm{i}h}{2m}\triangle u^*+\frac{ie^2}{hr}u^*)= \frac{\mathrm{i}h}{2m} (u^*\triangle u -u\triangle u^*) \]
			Since, \[\nabla \cdot (\nabla u \cdot u^* -u\cdot \nabla u^*)= \triangle u \cdot u^*+ \nabla u\cdot \nabla u^*- \nabla u\cdot \nabla u^*- u\cdot\triangle u^*=\triangle u \cdot u^*-u\cdot\triangle u^*,\]
			Therefore, \[ \frac{d}{dt}\int \left | u^2 \right | d \vec{x }=\int(u_{t}u^*+uu_{t}^*)  d \vec{x } = \frac{\mathrm{i}h}{2m}\int(u^*\triangle u -u\triangle u^*)d \vec{x }=\frac{\mathrm{i}h}{2m}\int \nabla \cdot (\nabla u \cdot u^* -u\cdot \nabla u^*)d \vec{x }\]
            In $\mathrm{R}^3$ space, let $B(r)$ be a ball with origin center and radius r.
			Then \[ \frac{d}{dt}\int_{\mathrm{R^3} }^{}  \left | u^2 \right | d \vec{x }= \lim_{r \to \infty}\int_{B(r)}\frac{d}{dt} \left | u^2 \right | d \vec{x }= \frac{\mathrm{i}h}{2m}\lim_{r \to \infty}\iiint_{r}^{} \nabla \cdot (\nabla u \cdot u^* -u\cdot \nabla u^*)d \vec{x }.\]
			From divergence theorem,
			\[\iiint_{r}^{} \nabla \cdot (\nabla u \cdot u^* -u\cdot \nabla u^*)dx dy dz= \iint_{\partial B(r)}^{}(\nabla u \cdot u^* -u\cdot \nabla u^*) \cdot \bar{n} dS.	\]
			Assume that $u$ and $\nabla u \to 0$ fast enough as $\left | \vec{x} \right | \to 0$, then we get
		
		\[ \lim_{r \to \infty}\iiint_{r}^{} \nabla \cdot (\nabla u \cdot u^* -u\cdot \nabla u^*)d \vec{x } =0,\]

		Therefore, \[ \frac{d}{dt}\int \left | u^2 \right | d \vec{x }=\frac{\mathrm{i}h}{2m}\lim_{r \to \infty}\iiint_{r}^{} \nabla \cdot (\nabla u \cdot u^* -u\cdot \nabla u^*)d \vec{x }=0,\]
			which means that at all later times, $ \int \left | u^2 \right | d \vec{x } =1$.
		


	\end{enumerate}
\item Strauss section 1.4
	\begin{enumerate}
		\item Exercise 1.
		\item Exercise 2.
	\end{enumerate}
\item Strauss section 1.5
	\begin{enumerate}
		\item Exercise 1.
		\item Exercise 4.
	\end{enumerate}
\end{enumerate}

\end{document}
