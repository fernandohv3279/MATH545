\documentclass{article}
\usepackage{amsmath}
\usepackage{mathrsfs}
\title{MATH 545: Homework 1}
\author{Yingli, Laurette, Fernando}
\date{October 9, 2023}
\begin{document}
\maketitle
\section{Section 2.4}
\subsection{Exercise 1}
Solve the diffusion equation with initial condition
\[
\phi(x)=1\text{ for }|x|<l \quad \text{and} \quad \phi(x)=0 \text{
for } |x|>l.
\]
Write your answer in terms of $\mathscr{E}$rf($x$).

\textbf{Solution:}

We saw in class that the solution to this problem is:
\[
    u(t,x)=S(t,\cdot)\ast u_0(x),
\]
Where $S$ is the heat kernel. So:
\begin{align*}
    u(t,x)
    &=S(t,\cdot)\ast u_0(x)\\
    &=\int S(t,x-y)u_0(y)dy\\
    &=\int_{|x|<l} S(t,x-y)u_0(y)dy+ \int_{|x|> l} S(t,x-y)u_0(y)dy\\
    &=\int_{|x|<l} S(t,x-y)dy\\
    &=\int_{-l}^lS(t,x-y)dy\\
    &=\int_{-l}^l\frac{1}{\sqrt{4\pi t D}} e^{-(x-y)^2/(4tD)}dy
\end{align*}
Now let $p=\frac{y-x}{\sqrt{4tD}}$, then $dp=\frac{dy}{\sqrt{4tD}}\iff
\sqrt{4tD}dp=dy$. So:
\begin{align*}
    u(t,x)
    &=\frac{\sqrt{4tD}}{\sqrt{4\pi t
    D}}\int_{\frac{-l-x}{\sqrt{4tD}}}^{\frac{l-x}{\sqrt{4tD}}} e^{-p^2}dp\\
    &=\frac{1}{\sqrt{\pi}}\int_{\frac{-l-x}{\sqrt{4tD}}}^{\frac{l-x}{\sqrt{4tD}}} e^{-p^2}dp\\
    &=\frac{1}{\sqrt{\pi}}\left(\int_{\frac{-l-x}{\sqrt{4tD}}}^{0}
    e^{-p^2}dp+\int_{0}^{\frac{l-x}{\sqrt{4tD}}} e^{-p^2}dp\right)\\
    &=\frac{1}{\sqrt{\pi}}\left(-\int_{0}^{\frac{-l-x}{\sqrt{4tD}}}
    e^{-p^2}dp+\int_{0}^{\frac{l-x}{\sqrt{4tD}}} e^{-p^2}dp\right)\\
    &=\frac{1}{2}\left(-\mathscr{E}\text{rf}\left(\frac{-l-x}{\sqrt{4tD}}\right)+\mathscr{E}\text{rf}\left(\frac{l-x}{\sqrt{4tD}}\right)\right).
\end{align*}
\subsection{Exercise 15}
Prove the uniqueness of the diffusion problem with Neumann boundary conditions:
\[
    u_t-ku_{xx}=f(x,t) \quad \text{for } 0<x<l,t>0 \quad u(x,0)=\phi(x)
\]
\[
    u_x(0,t)=g(t)\quad u_x(l,t)=h(t)
\]
by the energy method.

\textbf{Solution:}

Suppose there are two solutions: $u_1$ and $u_2$, now consider their
difference: $w=u_1-u_2$; it satisfies:
\[
    w_t -kw_{xx}=0 \quad \text{for } 0<x<l,t>0 \quad w(x,0)=0
\]
\[
w_x(0,t)=0\quad w_x(l,t)=0
\]
then:
\[
0=w_t-kw_{xx}=w(w_t-kw_{xx})=(w^2/2)_t+(-kww_x)_x+kw_x^2.
\]
Now we integrate with respect to $x$ from 0 to $l$ to get:
\[
0=\int_0^l\left(\frac{1}{2}w^2\right)_tdx-kww_x\Big|_{x=0}^{x=l}+k\int_0^lw_x^2dx.
\]
Because of the Neumann boundary condition we have $w_x(0,t)=w_x(l,t)=0$. So:
\[
\int_0^l\left(\frac{1}{2}w^2\right)_tdx=-k\int_0^lw_x^2dx\leq 0.
\]
Under the appropriate conditions (which we will assume to have) we can write:
\[
\int_0^l\left(\frac{1}{2}w^2\right)_tdx=\frac{d}{dt}\int_0^l\frac{1}{2}w^2dx
\]
Then:
\[
\frac{d}{dt}\int_0^l\frac{1}{2}w^2dx=-k\int_0^lw_x^2dx\leq 0.
\]
Which means that $\int_0^lw^2dx$ is decreasing with respect to $t$, so
\[
\int_0^l(w(x,t))^2dx\leq\int_0^l(w(x,0))^2dx
\]
The right hand side of the previous inequality is 0 because $w(x,0)=0$, it then
follows that $w(x,t)\equiv 0$, hence $u_1\equiv u_2$.
\subsection{Exercise 16}
Solve the diffusion equation with variable dissipation:
\[
    u_t-ku_{xx}+bu=0\quad \text{for }-\infty<x<\infty \quad \text{with } u(x,0)=\phi(x),
\]
where $b>0$ is a constant.

\textbf{Solution:}

As suggested by the hint we do the change of variables $u(x,t)=e^{-bt}v(x,t)$,
notice that with this change of variables the initial condition remains the
same, after the change we obtain:
\begin{align*}
-be^{-bt}v(x,t)+e^{-bt}v_t(x,t)-ke^{-bt}v_{xx}(x,t)+be^{-bt}v(x,t)=0\\
v_t(x,t)-kv_{xx}(x,t)=0\\
\end{align*}
Then our solution (using the notation we saw in class) is:
\[
v(x,t)=S(t,\cdot)\ast \phi(x)
\]
Finally we multiply by $e^{-bt}$ to recover $u$:
\[
    u(x,t)=e^{-bt}S(t,\cdot)\ast \phi(x).
\]
\subsection{Exercise 18}
Solve the heat equation with convection:
\[
u_t-ku_{xx}+Vu_x=0\quad \text{for }-\infty<x<\infty \quad \text{with } u(x,0)=\phi(x),
\]
where V is a constant.

\textbf{Solution:}

We proceed according to the hint and make the change $y=x-Vt$, let
$w(y,t)=u(y+Vt,t)=u(x,t)$, then $w_y=u_x$, $w_{yy}=u_{xx}$ and $w_t=Vu_x+u_t$.
Notice that the initial condition remains the same.
Substituting in the original equation we get
\begin{align*}
w_t-Vw_y -kw_{yy}+Vw_y=0\\
w_t-kw_{yy}=0.
\end{align*}
The solution to this is
\[
    w(y,t)=S(t,\cdot)\ast \phi(y),
\]
so
\[
    u(x,t)=S(t,\cdot)\ast \phi(x-Vt).
\]
\section{Section 4.1}
\subsection{Exercise 3}
A quantum-mechanical particle on the line with an infinite potential outside
the interval $(0,l)$ ("particle in a box") is given by Schr\"odinger's equation
$u_t=iu_{xx}$ on $(0,l)$ with Dirichlet conditions at the ends. Separate the
variables and use (8) to find its representation as a series.

\textbf{Solution:}

hola
\section{Section 4.2}
\subsection{Exercise 2}
\section{Section 4.3}
\subsection{Exercise 2}
\end{document}
