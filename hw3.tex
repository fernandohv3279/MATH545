\documentclass{article}
\usepackage{amsmath}
\usepackage{mathrsfs}
\title{MATH 545: Homework 1}
\author{Yingli, Laurette, Fernando}
\date{October 9, 2023}
\begin{document}
\maketitle
\section{Section 2.4}
\subsection{Exercise 1}
Solve the diffusion equation with initial condition
\[
\phi(x)=1\text{ for }|x|<l \quad \text{and} \quad \phi(x)=0 \text{
for } |x|>l.
\]
Write your answer in terms of $\mathscr{E}$rf($x$).

\textbf{Solution:}

We saw in class that the solution to this problem is:
\[
    u(t,x)=S(t,\cdot)\ast u_0(x),
\]
Where $S$ is the heat kernel. So:
\begin{align*}
    u(t,x)
    &=S(t,\cdot)\ast u_0(x)\\
    &=\int S(t,x-y)u_0(y)dy\\
    &=\int_{|x|<l} S(t,x-y)u_0(y)dy+ \int_{|x|> l} S(t,x-y)u_0(y)dy\\
    &=\int_{|x|<l} S(t,x-y)dy\\
    &=\int_{-l}^lS(t,x-y)dy\\
    &=\int_{-l}^l\frac{1}{\sqrt{4\pi t D}} e^{-(x-y)^2/(4tD)}dy
\end{align*}
Now let $p=\frac{y-x}{\sqrt{4tD}}$, then $dp=\frac{dy}{\sqrt{4tD}}\iff
\sqrt{4tD}dp=dy$. So:
\begin{align*}
    u(t,x)
    &=\frac{\sqrt{4tD}}{\sqrt{4\pi t
    D}}\int_{\frac{-l-x}{\sqrt{4tD}}}^{\frac{l-x}{\sqrt{4tD}}} e^{-p^2}dp\\
    &=\frac{1}{\sqrt{\pi}}\int_{\frac{-l-x}{\sqrt{4tD}}}^{\frac{l-x}{\sqrt{4tD}}} e^{-p^2}dp\\
    &=\frac{1}{\sqrt{\pi}}\left(\int_{\frac{-l-x}{\sqrt{4tD}}}^{0}
    e^{-p^2}dp+\int_{0}^{\frac{l-x}{\sqrt{4tD}}} e^{-p^2}dp\right)\\
    &=\frac{1}{\sqrt{\pi}}\left(-\int_{0}^{\frac{-l-x}{\sqrt{4tD}}}
    e^{-p^2}dp+\int_{0}^{\frac{l-x}{\sqrt{4tD}}} e^{-p^2}dp\right)\\
    &=\frac{1}{2}\left(-\mathscr{E}\text{rf}\left(\frac{-l-x}{\sqrt{4tD}}\right)+\mathscr{E}\text{rf}\left(\frac{l-x}{\sqrt{4tD}}\right)\right).
\end{align*}
\subsection{Exercise 15}
Prove the uniqueness of the diffusion problem with Neumann boundary conditions:
\[
    u_t-ku_{xx}=f(x,t) \quad \text{for } 0<x<l,t>0 \quad u(x,0)=\phi(x)
\]
\[
    u_x(0,t)=g(t)\quad u_x(l,t)=h(t)
\]
by the energy method.
\subsection{Exercise 16}
\subsection{Exercise 18}
\section{Section 4.1}
\subsection{Exercise 3}
\section{Section 4.2}
\subsection{Exercise 2}
\section{Section 4.3}
\subsection{Exercise 2}
\end{document}
