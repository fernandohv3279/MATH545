\documentclass[12pt]{article}%
\usepackage{amsfonts}
\usepackage{fancyhdr}
\usepackage{comment}
\usepackage{mathrsfs}
\usepackage[a4paper, top=2.5cm, bottom=2.5cm, left=2.2cm, right=2.2cm]%
{geometry}
\usepackage{times}
\usepackage{amsmath}
%\usepackage{changepage}
\usepackage{amssymb}
%\usepackage{graphicx}%
\setcounter{MaxMatrixCols}{30}
\newtheorem{theorem}{Theorem}
\newtheorem{acknowledgement}[theorem]{Acknowledgement}
\newtheorem{algorithm}[theorem]{Algorithm}
\newtheorem{axiom}{Axiom}
\newtheorem{case}[theorem]{Case}
\newtheorem{claim}[theorem]{Claim}
\newtheorem{conclusion}[theorem]{Conclusion}
\newtheorem{condition}[theorem]{Condition}
\newtheorem{conjecture}[theorem]{Conjecture}
\newtheorem{corollary}[theorem]{Corollary}
\newtheorem{criterion}[theorem]{Criterion}
\newtheorem{definition}[theorem]{Definition}
\newtheorem{example}[theorem]{Example}
\newtheorem{exercise}[theorem]{Exercise}
\newtheorem{lemma}[theorem]{Lemma}
\newtheorem{notation}[theorem]{Notation}
\newtheorem{problem}[theorem]{Problem}
\newtheorem{proposition}[theorem]{Proposition}
\newtheorem{remark}[theorem]{Remark}
\newtheorem{solution}[theorem]{Solution}
\newtheorem{summary}[theorem]{Summary}
\newenvironment{proof}[1][Proof]{\textbf{#1.} }{\ \rule{0.5em}{0.5em}}

\newcommand{\Q}{\mathbb{Q}}
\newcommand{\R}{\mathbb{R}}
\newcommand{\C}{\mathbb{C}}
\newcommand{\Z}{\mathbb{Z}}

\newcommand{\E}{\mathrm{E}}
\newcommand{\Var}{\mathrm{Var}}
\newcommand{\Cov}{\mathrm{Cov}}

\begin{document}

\title{Math 545 - PDE - Homework 4}
\author{YingLi, Fernando, Laurette }
\date{\today}
\maketitle

\section*{Section 6.1}
\subsection*{Exercise 2} Find the solutions that depend only on $r$ of the equation $u_{xx}+u_{yy}+u_{zz}=k^2u$, where $k$ is a positive constant.\\
\textbf{Solution:} 
The solutions depend only on $r$, the equation is $\triangle u =u_{rr} +\frac{2}{r}u_{r}=k^2u$.

Substitute $u=\frac{v}{r}$, then $u_{r}=\frac{v_{r}}{r}-\frac{v}{r^2}$, $u_{rr}=\frac{v_{rr}}{r}-\frac{2v_{r}}{r^2}+\frac{2v}{r^3}$.

Plug theses in the equation $u_{rr}+\frac{2}{r}u_{r}=k^2u \implies v_{rr}-k^2v=0.$

Solve the equation, we obtain $v=Acosh kr+Bsinh kr$, $A$ and $B$ are constants. 

Hence, $u=A\frac{cosh kr}{r}+B\frac{sinh kr}{r}$.

\subsection*{Exercise 4} Solve $u_{xx}+u_{yy}+u_{zz}=0$ in the spherical shell $0<a<r<b$ with the boundary conditions $u=A$ on $r=a$ and $u=B$ on $r=b$,
    where $A$ and $B$ are constants.\\
    \textbf{Solution:}
From the shape of domain and boundary conditions, the solution depends only on $r$, thus,

$\triangle u = u_{xx}+u_{yy}+u_{zz}= u_{rr} +\frac{2}{r}u_{r}=0$.

Solve this equation, we obtain $d(r^2u_{r})=0 \implies u_{r}=\frac{C}{r^2}$.

Intergrating about $r$, $u(r)=-\frac{C}{r}+D$, where $C$ and $D$ are constants.

From boudnary conditions, $u(a)=-\frac{C}{a}+D=A$, and $u(b)=-\frac{C}{b}+D=B$, thus $C=\frac{B-A}{b-a}ab$, $D=A+b\frac{B-A}{b-a}$.

Therefore, $u=\frac{B-A}{r(b-a)}ab+A+b\frac{B-A}{b-a}.$

    \subsection*{Exercise 9} A spherical shell with inner radius 1 and outer radius 2 has a steady-state temperature distribution. Its inner boundary is held at $100 ^{\circ} C$.
     Its outer boundary satisfies $\frac{\partial u}{\partial r}=-\gamma<0$, where $\gamma$ is a constant.
     \subsubsection*{Part a} Find the temperature.\\
     \textbf{Solution:}
     The temperature depends only on the radius, $\triangle u = u_{rr} +\frac{2}{r}u_{r}=0$, solve this equation, $u_{r}=\frac{C}{r^2}$ and $u(r)=-\frac{C}{r}+D$, where $C$ and $D$ are constants.

     From boundary conditions, $u(1)=-C+D=100$, $\frac{\partial u}{\partial r}|_{r=2}=\frac{C}{4}=-\gamma$, thus $C=-4\gamma, D=100-4\gamma$.

     Hence, $u(r)=\frac{4\gamma}{r}+100-4\gamma$.

     \subsubsection*{Part b} What are the hottest and coldest temperatures?\\
     \textbf{Solution:}
     From maximun principle, the maximun and minimun are only attained in the boundary.

     Thus, the coldest temperature is the minimum of $\frac{4\gamma}{r}+100-4\gamma $ for $r\in[1,2]$, i.e. $u_{min}=100-2\gamma$.

     Therefore, the hottest temperature is $100 ^{\circ} C$, in the inner boundary; the coldest temperature is $100-2\gamma$.

     \subsubsection*{Part c} Can you choose $\gamma$ so that the temperature on its outer boundary is $20 ^{\circ} C$.\\
     \textbf{Solution:}
      On the outer boundary, $u=100-2\gamma$, let $u=20$, implies $\gamma = 40$, so that choose $\gamma = 40$, the temperature on its outer boundary is $20 ^{\circ} C$.


     \section*{Section 6.2}
     \subsection*{Exercise 3} Find the harmonic function $u(x,y)$ in the square $D={0<x<\pi, 0<y<\pi}$ with the boundary conditions:
   
     $ u_{y}=0$, for $y=0$ and for $y=\pi$, $ u=0$  for  $x=0 $  and  $u=\cos ^{2} y=\frac{1}{2}(1+\cos 2 y)$  for $ x=\pi $.\\
     \textbf{Solution:}
     Assume $u(x,y)=X(x)Y(y)$. 
     
     By separating variables, $\frac{X''}{X}=-\frac{Y''}{Y}=\lambda$, the boundary conditions are $X(0)=Y'(0)=Y'(\pi)=0.$
    
    First, solve the equation $-\frac{Y''}{Y}=\lambda$ with boundary conditions $Y'(0)=0$ and $Y'(\pi)=0$,
    the solution is $Y_{n}(y)=C_{n}cos (ny), n\ge 1$.

    Then, solve the equation $\frac{X''}{X}=\lambda$ with boundary condition $X(0)=0$, the solution is 
    $X_{n}(x)=\widetilde{A_{n}}sinh (nx)$.
    For $\lambda = 0$, $X(x)=A_{0}x$.
    
    Thus, $u(x,y)=A_{0}x+\sum_{n=1}^{\infty}A_{n}sinh(nx)cosny$, where $A_{n}=\widetilde{A_{n}}C_{n}$.

    From the final boundary condition, \[u(\pi,y)=A_{0}\pi+A_{2}sinh(2x)cos2y=\frac{1}{2}(1+\cos 2 y),\]
    implies that \[A_{0}=\frac{1}{2\pi}, A_{2}=\frac{1}{2sinh(2\pi)}, A_{n}=0, \text{ for } n \ne 0, 2.\]

    Therefore, $u(x,y)=\frac{1}{2\pi}x+\frac{1}{2sinh(2\pi)}sinh(2x)cos2y.$


    \section*{Section 6.3}
    \subsection*{Exercise 1} Suppose that $u$ is a harmonic function in the disk $D=\{r<2\}$ and that $u=3 sin2 \theta +1$ for $r=2$. Without finding the solution, answer the following questions.\\
    \subsubsection*{Part a} Find the maximun value of $u$ in $\overline{D} $.\\
    \textbf{Solution:}
    From maximun principle, the maximun is only attained in the boundary. 
    For the disk D, the boundary is $r=2$, thus the maximun is attained when $u=3 sin2 \theta +1$ is the maximun.
    When $\theta = \frac{\pi}{4}, u_{max}=4$.
    
    Thus the maximun value of $u$ in $\overline{D}$ is 4.

    \subsubsection*{Part b} Calculate the value of $u$ at the orgin.\\
    \textbf{Solution:}
    By mean value property, the value of $u$ at the center of $D$ equals the average of $u$ on its circumference.
    
    Hence, $u(0)=\frac{1}{2\pi}\int_{0}^{2\pi}(3sin 2\theta+1)d\theta=1$.


    \section*{Section 6.4}
    \subsection*{Exercise 1} Solve $u_{xx}+u_{yy}=0$ in the exterior $\{r>a\}$ of a disk, with the boundary condition $u=1+3sin \theta$ on $ r=a$, and the conditon at infinity that $u$ be bounded as $r\to \infty$.\\
    \textbf{Solution:}
        From textbook, the general solution of $u$ is \[u(r, \theta)=\frac{1}{2} A_{0}+\sum_{n=1}^{\infty} r^{-n}\left(A_{n} \cos n \theta+B_{n} \sin n \theta\right).\] 
        The boundary condition means 
        \[u(a, \theta)=\frac{1}{2} A_{0}+\sum_{n=1}^{\infty} a^{-n}\left(A_{n} \cos n \theta+B_{n} \sin n \theta\right)=1+3sin \theta.\] 
        Thus, $A_{0}=2, B_{1}=3a, A_{n}=0, \text{ for } n \ge 1, B_{n}=0, \text{ for } n \ne 1.$
        
        Therefore, $u(r, \theta)=1+\frac{3a}{r}sin\theta.$
 

\end{document}
