\documentclass[12pt]{article}%
\usepackage{amsfonts}
\usepackage{fancyhdr}
\usepackage{comment}
\usepackage[a4paper, top=2.5cm, bottom=2.5cm, left=2.2cm, right=2.2cm]%
{geometry}
\usepackage{times}
\usepackage{amsmath}
\usepackage{changepage}
\usepackage{amssymb}
\usepackage{graphicx}%
\setcounter{MaxMatrixCols}{30}
\newtheorem{theorem}{Theorem}
\newtheorem{acknowledgement}[theorem]{Acknowledgement}
\newtheorem{algorithm}[theorem]{Algorithm}
\newtheorem{axiom}{Axiom}
\newtheorem{case}[theorem]{Case}
\newtheorem{claim}[theorem]{Claim}
\newtheorem{conclusion}[theorem]{Conclusion}
\newtheorem{condition}[theorem]{Condition}
\newtheorem{conjecture}[theorem]{Conjecture}
\newtheorem{corollary}[theorem]{Corollary}
\newtheorem{criterion}[theorem]{Criterion}
\newtheorem{definition}[theorem]{Definition}
\newtheorem{example}[theorem]{Example}
\newtheorem{exercise}[theorem]{Exercise}
\newtheorem{lemma}[theorem]{Lemma}
\newtheorem{notation}[theorem]{Notation}
\newtheorem{problem}[theorem]{Problem}
\newtheorem{proposition}[theorem]{Proposition}
\newtheorem{remark}[theorem]{Remark}
\newtheorem{solution}[theorem]{Solution}
\newtheorem{summary}[theorem]{Summary}
\newenvironment{proof}[1][Proof]{\textbf{#1.} }{\ \rule{0.5em}{0.5em}}

\newcommand{\Q}{\mathbb{Q}}
\newcommand{\R}{\mathbb{R}}
\newcommand{\C}{\mathbb{C}}
\newcommand{\Z}{\mathbb{Z}}

\newcommand{\E}{\mathrm{E}}
\newcommand{\Var}{\mathrm{Var}}
\newcommand{\Cov}{\mathrm{Cov}}

\begin{document}

\title{Math 545 - PDEs - Homework 2}
\author{YingLi Li, Fernando Herrera Valverde, Laurette Hamlin}
\date{\today}
\maketitle

\begin{enumerate}
    \item Strauss Section 1.5 \smallskip
    \begin{enumerate}
        \item Exercise 2 \smallskip \\
        Consider the problem
        \begin{align*}
            \frac{d^2u}{dx^2} + u = 0 \\
            u\left(0\right)=0 \text{ and } u\left(L\right)=0 \text{,} \\
        \end{align*} 
        consisting of an ODE and a pair of boundary conditions.  Clearly, the function $u\left(x\right) \equiv 0$ is a solution. Is this solution \emph{unique, or not}? Does the answer depend on $L$? \\
        Solution: \\
        Assume another solution $v$ exists.  Then $v'' + v' = 0$ 


        
        \item Exercise 3 \smallskip \\
        Solve the boundary problem $u'' = 0$ for $0 < x < 1$ with $u'\left(0\right) + ku\left(0\right) = 0$ and $u'\left(1\right) \pm ku\left(1\right) = 0$.  Do the $+$ and $-$ cases separately.  What is special about the case $k=2$? \\
        Solution: \\



        
    \end{enumerate}

    \pagebreak
    
    \item Strauss Section 1.6 \smallskip
    \begin{enumerate}
        \item Exercise 1 \smallskip \\
        What is the type of each of the following equations?
        \begin{enumerate}
            \item $u_{xx} - u_{xy} + 2u_y  u_{yx} + 4u = 0$. \smallskip \\

            
            \item $9u_{xx} + 6u_{xy} + u_{yy} + u_x = 0$. \smallskip \\

            
        \end{enumerate}
        \item Exercise 6 \smallskip \\
        Consider the equation $3u_y + u_{xy} =0$.
        \begin{enumerate}
            \item What is its type? \smallskip \\

            
            \item Find the general solution. (\emph{Hint:} Substitute $v = u_y$.) \smallskip \\

            
            \item With the auxiliary conditions $u\left(x,0\right) = e^{-3x}$ and $u_y\left(x,0\right)$, does a solution exist? Is it unique? \smallskip \\

            
        \end{enumerate}
    \end{enumerate}

    \pagebreak
    
    \item Strauss Section 2.1 \smallskip
    \begin{enumerate}
        \item Exercise 1 \smallskip \\
        Solve $u_{tt} = c^2u_{xx}$, $u\left(x,0\right) = e^x$, $u_t\left(x,0\right) = \sin{x}$. \smallskip \\
        
        \item Exercise 8 \smallskip \\
        A \emph{sperical wave} is a solution of the three-dimensional wave equation of the form $u\left(r,t\right)$ where $r$ is the distance to the origin (The spherical coordinate). The wave equation takes the form
        $$u_{tt} = c^2\left(u_{rr} + \frac{2}{r}u_r\right) \text{    ("spherical wave equation").}$$
        \begin{enumerate}
            \item Change variables $v= ru$ to get the equation for $v$:  $v_{tt} = c^2v_{rr}$. \smallskip \\

            \item Solve for $v$ using (3) and thereby solve the spherical wave equation. \smallskip \\

            \item Use (8) to solve it with initial conditions $u\left(r,0\right) = \phi\left(r\right)$, $u_t\left(r,0\right) = \psi\left(r\right)$, taking both $\phi\left(r\right)$ and $\psi\left(r\right)$ to be even functions of $r$. \smallskip \\

            
        \end{enumerate}
        
        \item Exercise 9 \smallskip \\
        Solve $u_{xx} - 3 u_{xt} - 4u_{tt} = 0$, $u\left(x,0\right) = x^2$, $u_t\left(x,0\right)$ = $e^x$. (\emph{Hint:} Factor the operator as we did for the wave equation.) \smallskip \\

        
    \end{enumerate}

    \pagebreak
    
    \item Strauss Section 2.2 \smallskip
    \begin{enumerate}
        \item Exercise 1 \smallskip \\
        Use the energy conservation of the wave equation to prove that the only solution with $\phi \equiv 0$ and $\psi \equiv 0$ is $u \equiv 0$. (\emph{Hint:} Use the first vanishing theorem in Section A.1.) \smallskip \\

        
        \item Exercise 2 \smallskip \\
        For a solution $u\left(x,t\right)$ of the wave equation with $\rho = T = c = 1$, the energy density is defined as $e = \frac{1}{2}\left(u_t^2 + u_x^2\right)$ and the momentum density as $p = _tu_x$.
        \begin{enumerate}
            \item Show that $\frac{\partial e}{\partial t} = \frac{\partial p}{\partial x}$ and $\frac{\partial p}{\partial t} = \frac{\partial e}{\partial x}$. \smallskip \\

            \item Show that both $e\left(x,t\right)$ and $p\left(x,t\right)$ also satisfy the wave equation. \smallskip \\
            
        \end{enumerate}
    \end{enumerate}
    
    \pagebreak

    \item Strauss Section 2.3 \smallskip
    \begin{enumerate}
        \item Exercise 1 \smallskip \\
        Consider the solution $1 - x^2 -2kt$ of the diffusion equation.  Find the locations of its maximum and its minimum in the closed rectangle $\{0 \leq x \leq 1, 0 \leq t \leq T\}$. \smallskip \\

        
    \end{enumerate}
\end{enumerate}



\end{document}
