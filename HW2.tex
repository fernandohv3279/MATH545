\documentclass[12pt]{article}%
\usepackage{amsfonts}
\usepackage{fancyhdr}
\usepackage{comment}
\usepackage[a4paper, top=2.5cm, bottom=2.5cm, left=2.2cm, right=2.2cm]%
{geometry}
\usepackage{times}
\usepackage{amsmath}
\usepackage{changepage}
\usepackage{amssymb}
\usepackage{graphicx}%
\setcounter{MaxMatrixCols}{30}
\newtheorem{theorem}{Theorem}
\newtheorem{acknowledgement}[theorem]{Acknowledgement}
\newtheorem{algorithm}[theorem]{Algorithm}
\newtheorem{axiom}{Axiom}
\newtheorem{case}[theorem]{Case}
\newtheorem{claim}[theorem]{Claim}
\newtheorem{conclusion}[theorem]{Conclusion}
\newtheorem{condition}[theorem]{Condition}
\newtheorem{conjecture}[theorem]{Conjecture}
\newtheorem{corollary}[theorem]{Corollary}
\newtheorem{criterion}[theorem]{Criterion}
\newtheorem{definition}[theorem]{Definition}
\newtheorem{example}[theorem]{Example}
\newtheorem{exercise}[theorem]{Exercise}
\newtheorem{lemma}[theorem]{Lemma}
\newtheorem{notation}[theorem]{Notation}
\newtheorem{problem}[theorem]{Problem}
\newtheorem{proposition}[theorem]{Proposition}
\newtheorem{remark}[theorem]{Remark}
\newtheorem{solution}[theorem]{Solution}
\newtheorem{summary}[theorem]{Summary}
\newenvironment{proof}[1][Proof]{\textbf{#1.} }{\ \rule{0.5em}{0.5em}}

\newcommand{\Q}{\mathbb{Q}}
\newcommand{\R}{\mathbb{R}}
\newcommand{\C}{\mathbb{C}}
\newcommand{\Z}{\mathbb{Z}}

\newcommand{\E}{\mathrm{E}}
\newcommand{\Var}{\mathrm{Var}}
\newcommand{\Cov}{\mathrm{Cov}}

\begin{document}
\setlength{\parindent}{20pt}

\title{Math 545 - PDEs - Homework 2}
\author{YingLi Li, Fernando Herrera Valverde, Laurette Hamlin}
\date{\today}
\maketitle

\begin{enumerate}
    \item Strauss Section 1.5 \smallskip
    \begin{enumerate}
        \item Exercise 2 \smallskip \\
        Consider the problem
        \begin{align*}
            &u''\left(x\right) + u'\left(x\right) = f\left(x\right) \\
            &u'\left(0\right)=u\left(0\right)=\frac{1}{2}\left[ u'\left(l\right) + u\left(l\right) \right] \text{,} \\
        \end{align*} 
        with $f\left(x\right)$ a given function.
        \begin{enumerate}
            \item Is the solution \emph{unique}? Explain. \\
                No, the solution is not unique. Suppose two solutions $u^{\left(i\right)}$, $i \in \{1,2\}$ exist that both satisfy the initial conditions.  Then 
                \begin{align*}
                    &u^{\left(i\right)}_{xx} + u^{\left(i\right)}_{x} = f\left(x\right) \\
                    &u^{\left(i\right)}_{x}\left(x=0\right)=u^{\left(i\right)}\left(x=0\right)=\frac{1}{2}\left[ u^{\left(i\right)}_{x}\left(l\right) + u^{\left(i\right)}\left(l\right) \right]
                \end{align*}
                Applying a Laplacian operator with both solutions yields $\Delta u = u^1 - u^2$. By linearity,
                \begin{align*}
                    \Delta u &= u^1 - u^2 \\
                    &= \left(u^{\left(1\right)}_{xx} + u^{\left(1\right)}_{x}\right)
                    - \left( u^{\left(2\right)}_{xx} + u^{\left(2\right)}_{x}\right) \\
                    &= \left(u^{\left(1\right)}_{xx} - u^{\left(2\right)}_{xx}\right)
                    + \left(+ u^{\left(1\right)}_{x} - u^{\left(2\right)}_{x}\right) \\ 
                    &= u_{xx} + u_{x}  \text{ , where $u$ is a general solution}\\
                    &= f\left(x\right) - f\left(x\right) \\
                    &= 0.
                \end{align*} 
                Hence, \underline{the general solution to $u_{xx} + u_{x} = 0$ is $C_1e^{-x} + C_2$.}  \smallskip \\ \\
                To find the exact solution, we note that the initial conditions can also be transformed into $u^{\left(i\right)}_{x}\left(0\right)=u_{x}\left(0\right)=0$ and $u^{\left(i\right)}\left(0\right) = u\left(0\right) = 0$.  Then $u\left(0\right) = C_1 + C_2$ and $u'\left(0\right)=-C_1$.  Thus,
                \begin{align*}
                    &u'\left(0\right) = \frac{1}{2}\left[ u'\left(l\right) + u\left(l\right) \right] \\
                    &-C_1 = \frac{1}{2}\left[-C_1e^{-l} + C_1e^{-l} + C2\right] \\
                    &-C_1 = \frac{1}{2}C_2 \\
                    &\underline{C_2 = -2C_1}. \\
                \end{align*}
                Substituting this in to the initial condition for $u$, 
                \begin{align*}
                    &u\left(0\right) =\frac{1}{2}\left[u'\left(l\right) + u\left(l\right)\right] \\
                    &u\left(0\right) =\frac{1}{2}\left[-C_1e^{-l} + -2C_1 + C_1e^{-l}\right] \\ 
                    &u\left(0\right) = -C_1.
                \end{align*}
                The exact solution thus has a parameter, namely \underline{$u = -2C_1 + C_1e^{-x}.$} Therefore, there exists more than one unique solution.  \checkmark \smallskip \\
            
            \item Does a solution necessarily \emph{exist}, or is there a a condition that $f\left(x\right)$ must satisfy for existence? Explain. \\
                There are conditions $x$ to satisfy existence because the PDE is equal to a function $f$, so the domain must be considered.  Specifically,
                \begin{align*}
                    \int_0^l f\left(x\right)dx &= \int_0^l{u''\left(x\right) + u'\left(x\right)dx} \\
                    &= \left(u'\left(x\right)  + u\left(x\right)\right)|_0^l \\
                    &= u'\left(l\right)  + u\left(l\right).
                \end{align*}
                Hence, the existence of the solution depends upon $l$.  \checkmark
        \end{enumerate} 
        
        \pagebreak
        
        \item Exercise 3 \smallskip \\
        Solve the boundary problem $u'' = 0$ for $0 < x < 1$ with $u'\left(0\right) + ku\left(0\right) = 0$ and $u'\left(1\right) \pm ku\left(1\right) = 0$.  Do the $+$ and $-$ cases separately.  What is special about the case $k=2$? \\

        This is a one-dimensional problem with homogeneous \emph{Robin} boundary conditions on two endpoints ($x = 0$ and $x = 1$). We know we have existence and uniqueness in the solution, but stability is in question. We would like to know if the change in sign in the second boundary condition causes large changes to $u$. 

        The general solution to this PDE is $u\left(x,t\right) = C_1 f\left(t\right)x + C_2 g\left(t\right)$. For the first boundary condition, we have
        \begin{align*}
            &u'\left(0\right) + ku\left(0\right) = 0 \\
            &C_1 f\left(t\right) + k C_2 g\left(t\right) = 0 \\
            &C_1 f\left(t\right) = -k C_2 g\left(t\right) \\
        \end{align*}
        For the second boundary condition, 
        \begin{equation*}
            \begin{aligned}[c]
                u'\left(1\right) + ku\left(1\right) &= 0 \\
                C_1 f\left(t\right) + k \left(C_1 f\left(t\right) + C_2 g\left(t\right) \right) &= 0 \\
                C_1 f\left(t\right)\left(1+k\right)  + k C_2 g\left(t\right) &= 0 \\
                -k C_2 g\left(t\right) \left(1+k\right)  + k C_2  g\left(t\right) &= 0 \\
                k^2 C_2  g\left(t\right) &= 0 \\
                C_2 &= 0. \\
            \end{aligned}
            \qquad\Longleftrightarrow\qquad
            \begin{aligned}[c]
                u'\left(1\right) - ku\left(1\right) = 0 \\
                C_1 f\left(t\right) - k \left(C_1 f\left(t\right) + C_2 g\left(t\right) \right) &= 0 \\
                C_1 f\left(t\right)\left(1-k\right)  - k C_2 g\left(t\right) &= 0 \\
                -k C_2 g\left(t\right) \left(1-k\right)  - k C_2 g\left(t\right) &= 0 \\
                k^2 C_2  g\left(t\right) - 2k C_2 g\left(t\right) &= 0 \\
                k\left(k-2\right) &= 0. \\
            \end{aligned}
        \end{equation*}
        For the negative sign case, the extra term creates a situation where $C_2$ can have multiple values.  Thus the solution is not stable at $k=2$.   \checkmark \\
    \end{enumerate}

    \pagebreak
    
    \item Strauss Section 1.6 \smallskip
    \begin{enumerate}
        \item Exercise 1 \smallskip \\
        What is the type of each of the following equations?
        \begin{enumerate}
            \item $\underline{u_{xx}} - \underline{u_{xy}} + 2u_y + \underline{u_{yy}} - \underline{3u_{yx}} + 4u = 0$. \smallskip \\
            Using $\mathcal{D} = a_{12}^2 - a_{11}a_{12}$, we have
            $$\left(-\frac{4}{2}\right)^2 - \left(1\right)\left(1\right) > 0.$$
            Thus, the PDE is \underline{hyperbolic}. \\

            
            \item $\underline{9u_{xx}} + \underline{6u_{xy}} + \underline{u_{yy}} + u_x = 0$. \smallskip \\Again using $\mathcal{D} = a_{12}^2 - a_{11}a_{12}$, we have
            $$\left(\frac{6}{2}\right)^2 - \left(9\right)\left(1\right) = 0.$$
            Thus, the PDE is \underline{parabolic}. \\

            
        \end{enumerate}
        \item Exercise 6 \smallskip \\
        Consider the equation $3u_y + u_{xy} =0$.
        \begin{enumerate}
            \item What is its type? \smallskip \\
            Since $a_{12}^2 - a_{11}a_{12} = \left(\frac{1}{2}\right)^2 - \left(0\right)\left(0\right) > 0$, it is \underline{hyperbolic}.
            
            \item Find the general solution. (\emph{Hint:} Substitute $v = u_y$.) \smallskip \\
            If we substitute $v = u_y$, then $v_x = u_{xy}$ and the equation is simplified into the ODE $3v + v_x = 0$. This yields $u_y = f\left(y\right) e^{-3x}$ so that the general solution is found by
            \begin{align*}
                &\iint u\left(x,y\right) dydx = -\frac{e^{-3x}}{3} \int f\left(y\right) dy \\
                &\boxed{u\left(x,y\right) = -\frac{F\left(y\right)}{3}y e^{-3x} + g\left(x\right).} \\
            \end{align*}

            
            \item With the auxiliary conditions $u\left(x,0\right) = e^{-3x}$ and $u_y\left(x,0\right) = 0$, does a solution exist? Is it unique? \smallskip \\
            At $u\left(x,0\right)$, we see that \underline{$g\left(x\right) = e^{-3x}.$} \\
            At $u_y\left(x,0\right)$, we see that \underline{$f\left(y\right) = 0.$} \\            
        \end{enumerate}
        Given that we have a general solution in $f$ and $g$, we know that \underline{a solution exists}.  However, the boundary conditions do not solve for at most one solution.  Therefore \underline{we do not have uniqueness}.
    \end{enumerate}

    \pagebreak
    
    \item Strauss Section 2.1 \smallskip
    \begin{enumerate}
        \item Exercise 1 \smallskip \\
        Solve $u_{tt} = c^2u_{xx}$, $u\left(x,0\right) = e^x$, $u_t\left(x,0\right) = \sin{x}$. \smallskip \\

        This is a wave equation with general solution $u\left(x,t\right) = f\left(x+ct\right) + g\left(x-ct\right)$, indicating waves traveling to both the left and right.  To solve this wave equation with initial value functions $u\left(x,0\right)=\phi\left(x\right)$ and $u_t\left(x,0\right)=\psi\left(x\right)$, we use the solution formula from d'Alembert.
        \begin{align*}
            u\left(x,t\right) &= \frac{1}{2}\left[\phi\left(x+ct\right) + \phi\left(x-ct\right)\right] + \frac{1}{2c}\int_{x-ct}^{x+ct}\psi\left(s\right)ds \\
            &= \frac{1}{2}\left[e^{x+ct} + e^{x-ct}\right] + \frac{1}{2c}\int_{x-ct}^{x+ct}\sin{s}ds \\
            &= e^{x}\left[\frac{e^{ct} + e^{-ct}}{2}\right] + 
                \frac{1}{2c}\left[\cos{\left(x+ct\right)} -\cos{\left(x-ct\right)}\right] \\
            &= \boxed{e^{x}\cosh{\left(ct\right)} + 
                \frac{1}{c}\left[\sin{\left(x\right)}\sin{\left(ct\right)}\right]}. \\
        \end{align*}
        
        \item Exercise 8 \smallskip \\
        A \emph{spherical wave} is a solution of the three-dimensional wave equation of the form $u\left(r,t\right)$ where $r$ is the distance to the origin (the spherical coordinate). The wave equation takes the form
        $$u_{tt} = c^2\left(u_{rr} + \frac{2}{r}u_r\right) \text{    ("spherical wave equation").}$$
        \begin{enumerate}
            \item Change variables $v = ru$ to get the equation for $v$:  $v_{tt} = c^2v_{rr}$. \smallskip \\
            If we redefine $u = \frac{v}{r}$, then the spherical wave equation can be transformed. 
            \begin{align*}
                u_{tt} &= c^2\left(u_{rr} + \frac{2}{r}u_r\right) \\
                \left(\frac{v}{r}\right)_{tt} &= c^2\left[\left(\frac{v}{r}\right)_{rr} + \frac{2}{r}\left(\frac{v}{r}\right)_r\right] \\
                \left(\frac{v_{tt}}{r}\right) &= c^2\left[\left(\frac{v}{r}\right)_{rr} + \frac{2}{r}\left(\frac{rv_r - v}{r^2}\right)\right] \\
            \end{align*}

            \item Solve for $v$ using (3) and thereby solve the spherical wave equation. \smallskip \\

            \item Use (8) to solve it with initial conditions $u\left(r,0\right) = \phi\left(r\right)$, $u_t\left(r,0\right) = \psi\left(r\right)$, taking both $\phi\left(r\right)$ and $\psi\left(r\right)$ to be even functions of $r$. \smallskip \\

            
        \end{enumerate}
        
        \item Exercise 9 \smallskip \\
        Solve $u_{xx} - 3 u_{xt} - 4u_{tt} = 0$, $u\left(x,0\right) = x^2$, $u_t\left(x,0\right)$ = $e^x$. (\emph{Hint:} Factor the operator as we did for the wave equation.) \smallskip \\

        
    \end{enumerate}

    \pagebreak
    
    \item Strauss Section 2.2 \smallskip
    \begin{enumerate}
        \item Exercise 1 \smallskip \\
        Use the energy conservation of the wave equation to prove that the only solution with $\phi \equiv 0$ and $\psi \equiv 0$ is $u \equiv 0$. (\emph{Hint:} Use the first vanishing theorem in Section A.1.) \smallskip \\

        
        \item Exercise 2 \smallskip \\
        For a solution $u\left(x,t\right)$ of the wave equation with $\rho = T = c = 1$, the energy density is defined as $e = \frac{1}{2}\left(u_t^2 + u_x^2\right)$ and the momentum density as $p = _tu_x$.
        \begin{enumerate}
            \item Show that $\frac{\partial e}{\partial t} = \frac{\partial p}{\partial x}$ and $\frac{\partial p}{\partial t} = \frac{\partial e}{\partial x}$. \smallskip \\

            \item Show that both $e\left(x,t\right)$ and $p\left(x,t\right)$ also satisfy the wave equation. \smallskip \\
            
        \end{enumerate}
    \end{enumerate}
    \item Strauss Section 2.3 \smallskip
    \begin{enumerate}
        \item Exercise 1 \smallskip \\
        Consider the solution $1 - x^2 -2kt$ of the diffusion equation.  Find the locations of its maximum and its minimum in the closed rectangle $\{0 \leq x \leq 1, 0 \leq t \leq T\}$. \smallskip \\

        
    \end{enumerate}
\end{enumerate}



\end{document}
