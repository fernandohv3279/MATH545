\documentclass[12pt]{article}%
\usepackage{amsfonts}
\usepackage{fancyhdr}
\usepackage{comment}
\usepackage{mathrsfs}
\usepackage[a4paper, top=2.5cm, bottom=2.5cm, left=2.2cm, right=2.2cm]%
{geometry}
\usepackage{times}
\usepackage{amsmath}
%\usepackage{changepage}
\usepackage{amssymb}
\usepackage{amsthm}
%\usepackage{graphicx}%

\newcommand{\Q}{\mathbb{Q}}
\newcommand{\R}{\mathbb{R}}
\newcommand{\C}{\mathbb{C}}
\newcommand{\Z}{\mathbb{Z}}

\newcommand{\E}{\mathrm{E}}
\newcommand{\Var}{\mathrm{Var}}
\newcommand{\Cov}{\mathrm{Cov}}

\begin{document}

\title{Math 545 - PDE - Homework 4}
\author{YingLi, Fernando, Laurette }
\date{\today}
\maketitle

\section*{Section 7.1}
\subsection*{Exercise 3}
Derive the three-dimensional maximum principle from the mean value property.\\
\textbf{Solution:}\\
Let's denote the ball or radius $r$ and center $x$ by $B_r(x)$, and let $D$ be
open. The mean value property tells us that if $B_r(x)\subset D$ then
\[
    u(x)=\frac{1}{4\pi r^2}\int_{\partial B_r(x)}u.
\]
Now let's assume that $u$ attains it's maximum at $x_M\in D$, because $D$ is
open we can find $\varepsilon$ such that $B_\varepsilon(x_M) \subset D$ and
\begin{equation}\label{mvp}
    u(x_M) = \frac{1}{4\pi \varepsilon^2}\int_{\partial B_\varepsilon(x_M)} u.
\end{equation}
Notice that if at any point $x\in \partial B_\varepsilon(x_M)$ we have
\begin{equation}\label{ineq1}
u(x) < u(x_M)
\end{equation}
because $u$ is continuous then we would need another point
$y\in \partial B_\varepsilon(x_M)$ such that
\begin{equation}\label{ineq2}
u(y)>u(x_M)
\end{equation}
in order to satisfy
(\ref{mvp}). However by definition, for
any $x\in D$ and in particular any $x\in \partial
B_\varepsilon(x_M)$
\begin{equation}\label{trueineq}
    u(x)\leq u(x_M),
\end{equation}
so (\ref{ineq2}) is impossible and therefore (\ref{ineq1}) is impossible. This
in combination with (\ref{trueineq}) implies that
\[
u(x)=u(x_M) \quad \forall x \in \partial B_\varepsilon(x_M).
\]
Since this is true for all smaller $\varepsilon$'s we get that $u$ is constant
on $B_\varepsilon(x_M)$. Now we continue with the same argument on a different
ball that overlaps with $B_\varepsilon(x_M)$ and conclude that $u$ is constant
there. If $D$ is convex we can cover the domain with balls and conclude that
$u$ is constant in $D$. \qedsymbol
\section*{Section 7.2}
\subsection*{Exercise 2}
Let $\phi(x)$ be any $C^2$ function defined on all of the three-dimensional
space that vanishes outside some sphere. Show that
\[
    \phi(0)=-\int\int\int \frac{1}{|x|}\Delta \phi(x) \frac{dx}{4\pi}.
\]
The integration is taken over the region where $\phi(x)$ is not zero.\\
\textbf{Solution:}\\
Let $B_r$ be the ball of radius $r$ centered at the origin. For
$\varepsilon<r$. Let $D=B_r-B_\varepsilon$.\\
Let $v=-(4\pi|x|)^{-1}$. Using Green's identity we get:
\[
    \int_D \phi\Delta v  - v \Delta \phi = \int_{\partial D} \phi \partial_n v
    - v \partial_n \phi,
\]
Since $v$ is harmonic in $D$ we get
\[
    \int_D v \Delta \phi = -\int_{\partial D} \phi \partial_n v - v \partial_n \phi = -
    \int_{\partial B_r} \phi \partial_n v - v \partial_n \phi - \int_{\partial
    B_\varepsilon} \phi \partial_n v - v \partial_n \phi.
\]
Letting $r\to \infty$ and using the fact that $\phi$ vanishes we get
\[
    \int_{B_\epsilon^c}v\Delta\phi = \int_{\partial B_\varepsilon}
    v\partial_n \phi - \phi \partial_n v.
\]
Now we let $\varepsilon \to 0$. Notice that because $\phi$ is $C^2$, its
derivative is bounded so we get
\[
    \bigg| \int_{\partial B_\varepsilon} v\partial_n \phi\bigg|
    \leq \int_{\partial B_\varepsilon} |v\partial_n \phi|
    \leq C \int_{\partial B_\varepsilon}|v| =
    C/(4\pi\varepsilon)\int_{B_\varepsilon}1=C4\pi\varepsilon^2/(4\pi\varepsilon)
    = C\varepsilon,
\]
so now we have
\[
    \int_{\mathbb{R}^3}v\Delta\phi = -\lim_{\varepsilon\to0}\int_{\partial B_\varepsilon}\phi\partial_n v,
\]
and we observe that $\partial_n v=-\partial_r v= -1/(4\pi r^2)$ so on
$\partial B_\varepsilon$ we have $\partial_n v = -1/(4\pi\varepsilon^2)$, then
\begin{align*}
    \int_{\mathbb{R}^3}v\Delta\phi &=
    \lim_{\varepsilon\to0}\left[\frac{1}{4\pi\varepsilon^2}\int_{\partial
    B_\varepsilon}\phi\right]\\
    &=\phi(0). \quad \qedsymbol
\end{align*}
The last equality is true because $\frac{1}{4\pi
\varepsilon^2}\int_{\partial B_\varepsilon}\phi$ is the average value of $\phi$
on the sphere $\partial B_\varepsilon$ and $\phi$ is continuous (this is how it
is justified in the book).
\section*{Section 7.3}
\subsection*{Exercise 1}
Show that the Green's function is unique.\\
\textbf{Solution 1:}\\
Suppose $G_1$ and $G_2$ are Green's functions. Consider $G=G_1-G_2$. We have
\[
    \Delta_x G(x,y)=\Delta_x G_1(x,y)-\Delta_x G_2(x,y) = \delta(x-y) -
    \delta(x-y) =0.
\]
This means that $G$ is harmonic with respect to $x$. Since
$G_1=G_2$ on $\partial D$, we have $G=0$ on $\partial D$, and by the unicity of
the Dirichlet problem we have $G\equiv0$. $\qedsymbol$
\\\textbf{Solution 2:}\\
Suppose there are two Green's functions $G_1$ and $G_2$. Consider
\begin{align*}
    G=G_1-G_2&= G_1 + 1/(4\pi |x-x_0|) - [G_2 + 1/(4\pi |x-x_0|)]\\
\end{align*}
According to the book $G_i+1/(4\pi|x-x_o|)$ ($i=1,2$) is finite at $x_0$ and
has continuous second derivatives everywhere. Then the same is true for $G$.\\
\textbf{Upshot:} G has continuous second derivatives at every point in $D$.\\
Also we see that $\Delta G =0$ in \textbf{all} of $D$ (including $x_0$!),
because of the previous upshot.\\
Finally since $G=0$ on $\partial D$ (because $G_1=G_2$ on $\partial D$) and $G$
is harmonic, by the unicity of the Dirichlet problem we conclude that
$G\equiv0$. $\qedsymbol$
% TODO: Add second solution!!
\subsection*{Exercise 2}
Proof Theorem 2, which gives the solution of Poisson's equation in terms of the
Green's function.\\
\textbf{Solution:}\\
Suppose that u satisfies
\[
    \Delta u =f \quad \text{in D} \quad u=h \quad \text{on }\partial D,
\]
then
\[
    u(y)=\int_{\partial D}h(x)\partial_nG(x,y) + \int_D f(x)G(x,y)
\]
\textbf{Proof:}
Green's function satisfies:
\[
\begin{cases}
    \Delta_xG(x,y)=\delta(x-y) \quad \text{in } D\\
    G(x,y)=0 \qquad\qquad\quad x\in \partial D
\end{cases}
\]
Using integration by parts (divergence theorem) we get:
\begin{equation}\label{byParts}
    \int_D \Delta_x G(x,y)u(x)dx=\int_{\partial D} u(x)\partial_nG(x,y)ds - \int_D \nabla_xG(x,y)\nabla u(x)dx.
\end{equation}
For the second term notice that
\begin{align} \label{secondByParts}
    \int_D \nabla_xG(x,y)\nabla u(x)dx&=\int_{\partial D}G(x,y)\partial_nu(x)ds - \int_D G(x,y)\Delta u(x)dx \nonumber\\
    \int_D \nabla_xG(x,y)\nabla u(x)dx&=- \int_D G(x,y)\Delta u(x)dx \nonumber\\
    \int_D \nabla_xG(x,y)\nabla u(x)dx&=- \int_D G(x,y)f(x)dx.
\end{align}
Where we used the fact that $G$ vanishes at the boundary and that $\Delta u =f$.
Substituting (\ref{secondByParts}) into (\ref{byParts}) we get
\[
    \int_D \Delta_x G(x,y)u(x)dx=\int_{\partial D} u(x)\partial_nG(x,y)ds + \int_D f(x)G(x,y)dx.
\]
since $\Delta_xG(x,y)=\delta(x-y)$ and $u=h$ on $\partial D$ we get
\[
    u(y)=\int_{\partial D} h(x)\partial_nG(x,y)ds + \int_D f(x)G(x,y)dx. \quad \qedsymbol
\]
\section*{Section 7.4}
\subsection*{Exercise 1}
Find the one-dimensional Green's function for the interval $(0,l)$. The three
properties defining it can be restated as follows.\\
(i) It solves $G''=0$ for $x\neq x_0$ ("harmonic").\\
(ii) $G(0)=G(l)=0$.\\
(iii) $G(x)$ is continuous at $x_0$ and $G(x)+\frac{1}{2}|x-x_0|$ is harmonic at $x_0$.\\
\textbf{Solution:}\\
For $0\leq x \leq x_0\leq l$, we know $G(x)=ax+b$ from (i) and $b=0$ from (ii). So
\[
G(x)=ax.
\]
For $0\leq x_0\leq x\leq l$, we know again that $G(x)=cx+d$ from
(i) and $d=-cl$ from (ii). For $G$ to be continuous at $x_0$ we
need $ax_0=cx_0+d=cx_0-cl=c(x_0-l)$ and for $G$ to be harmonic it's second
derivative must be continuous at $x_0$ so we need $a-\frac{1}{2}=c+\frac{1}{2}$. Solving for $a$ and $c$ we find $c=-x_0/l$ and $a=-x_0/l+1$, hence
\[
    G(x) =\begin{cases}
        x(l-x_0)/l \quad \text{for } 0\leq x\leq x_0 \leq l\\
        x_0(l-x)/l \quad \text{for } 0\leq x_0\leq x \leq l
    \end{cases}. \quad \qedsymbol
\]
\subsection*{Exercise 5}
Notice that the function $xy$ is harmonic in the half-plane
$\{y>0\}$ and vanishes on the boundary line $\{y=0\}$. The function 0 has the same properties. Does this mean that the solution is not unique? Explain.\\
\textbf{Solution:}\\
Indeed these are two different functions that solve the Dirichlet
problem
\[
    \begin{cases}
    \Delta u =0 \quad \text{in } D\\
    u =0 \quad \text{on } \partial D,
    \end{cases}
\]
where $D$ is the upper half-plane. The reason for this is that the
Dirichlet problem has a unique solution when $D$ is a bounded domain,
which is not the case here. We would need to add more requirements to $u$
in order to guarantee uniqueness. $\qedsymbol$
\end{document}
