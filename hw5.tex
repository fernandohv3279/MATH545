\documentclass[12pt]{article}%
\usepackage{amsfonts}
\usepackage{fancyhdr}
\usepackage{comment}
\usepackage{mathrsfs}
\usepackage[a4paper, top=2.5cm, bottom=2.5cm, left=2.2cm, right=2.2cm]%
{geometry}
\usepackage{times}
\usepackage{amsmath}
%\usepackage{changepage}
\usepackage{amssymb}
\usepackage{amsthm}
%\usepackage{graphicx}%

\newcommand{\Q}{\mathbb{Q}}
\newcommand{\R}{\mathbb{R}}
\newcommand{\C}{\mathbb{C}}
\newcommand{\Z}{\mathbb{Z}}

\newcommand{\E}{\mathrm{E}}
\newcommand{\Var}{\mathrm{Var}}
\newcommand{\Cov}{\mathrm{Cov}}

\begin{document}

\title{Math 545 - PDE - Homework 4}
\author{YingLi, Fernando, Laurette }
\date{\today}
\maketitle

\section*{Section 7.1}
\subsection*{Exercise 3}
Derive the three-dimensional maximum principle from the mean value property.\\
\textbf{Solution:}\\
Let's denote the ball or radius $r$ and center $x$ by $B_r(x)$, and let $D$ be
open. The mean value property tells us that if $B_r(x)\subset D$ then
\[
    u(x)=\frac{1}{4\pi r^2}\int_{\partial B_r(x)}u.
\]
Now let's assume that $u$ attains it's maximum at $x_M\in D$, because $D$ is
open we can find $\varepsilon$ such that $B_\varepsilon(x_M) \subset D$ and
\begin{equation}\label{mvp}
    u(x_M) = \frac{1}{4\pi \varepsilon^2}\int_{\partial B_\varepsilon(x_M)} u.
\end{equation}
Notice that if at any point $x\in \partial B_\varepsilon(x_M)$ we have
\begin{equation}\label{ineq1}
u(x) < u(x_M)
\end{equation}
because $u$ is continuous then we would need another point
$y\in \partial B_\varepsilon(x_M)$ such that
\begin{equation}\label{ineq2}
u(y)>u(x_M)
\end{equation}
in order to satisfy
(\ref{mvp}). However by definition, for
any $x\in D$ and in particular any $x\in \partial
B_\varepsilon(x_M)$
\begin{equation}\label{trueineq}
    u(x)\leq u(x_M),
\end{equation}
so (\ref{ineq2}) is impossible and therefore (\ref{ineq1}) is impossible. This
in combination with (\ref{trueineq}) implies that
\[
u(x)=u(x_M) \quad \forall x \in \partial B_\varepsilon(x_M).
\]
Since this is true for all smaller $\varepsilon$'s we get that $u$ is constant
on $B_\varepsilon(x_M)$. Now we continue with the same argument on a different
ball that overlaps with $B_\varepsilon(x_M)$ and conclude that $u$ is constant
there. If $D$ is convex we can cover the domain with balls and conclude that
$u$ is constant in $D$. \qedsymbol
\section*{Section 7.2}
\subsection*{Exercise 2}
\section*{Section 7.3}
\subsection*{Exercise 1}
\subsection*{Exercise 2}
\section*{Section 7.4}
\subsection*{Exercise 1}
\subsection*{Exercise 5}
\end{document}
