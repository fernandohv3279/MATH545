\documentclass[12pt]{article}%
\usepackage{amsfonts}
\usepackage{fancyhdr}
\usepackage{comment}
\usepackage[a4paper, top=2.5cm, bottom=2.5cm, left=2.2cm, right=2.2cm]%
{geometry}
\usepackage{times}
\usepackage{amsmath}
\usepackage{changepage}
\usepackage{amssymb}
\usepackage{graphicx}%
\setcounter{MaxMatrixCols}{30}
\newtheorem{theorem}{Theorem}
\newtheorem{acknowledgement}[theorem]{Acknowledgement}
\newtheorem{algorithm}[theorem]{Algorithm}
\newtheorem{axiom}{Axiom}
\newtheorem{case}[theorem]{Case}
\newtheorem{claim}[theorem]{Claim}
\newtheorem{conclusion}[theorem]{Conclusion}
\newtheorem{condition}[theorem]{Condition}
\newtheorem{conjecture}[theorem]{Conjecture}
\newtheorem{corollary}[theorem]{Corollary}
\newtheorem{criterion}[theorem]{Criterion}
\newtheorem{definition}[theorem]{Definition}
\newtheorem{example}[theorem]{Example}
\newtheorem{exercise}[theorem]{Exercise}
\newtheorem{lemma}[theorem]{Lemma}
\newtheorem{notation}[theorem]{Notation}
\newtheorem{problem}[theorem]{Problem}
\newtheorem{proposition}[theorem]{Proposition}
\newtheorem{remark}[theorem]{Remark}
\newtheorem{solution}[theorem]{Solution}
\newtheorem{summary}[theorem]{Summary}
\newenvironment{proof}[1][Proof]{\textbf{#1.} }{\ \rule{0.5em}{0.5em}}

\newcommand{\Q}{\mathbb{Q}}
\newcommand{\R}{\mathbb{R}}
\newcommand{\C}{\mathbb{C}}
\newcommand{\Z}{\mathbb{Z}}

\newcommand{\E}{\mathrm{E}}
\newcommand{\Var}{\mathrm{Var}}
\newcommand{\Cov}{\mathrm{Cov}}

\begin{document}

\title{Math 545 - PDEs - Homework 2}
\author{YingLi Li, Fernando Herrera Valverde, Laurette Hamlin}
\date{\today}
\maketitle

\begin{enumerate}
    \item Strauss Section 1.5 \smallskip
    \begin{enumerate}
        \item Exercise 2 \smallskip \\
        Consider the problem
        \begin{align*}
            u''(x) + u(x) = f(x),\\
            u'(0)=u(0)=\frac{1}{2}[u'(l)+u(l)], 
        \end{align*} 
        with $f(x)$ a given function.
        (a) Is the solution unique? Explain.\\
     Solution: \\
       The solution is not unique. Since for ODE, the solution is special solution plus general solution.
       
       To be specific, since the ODE system is linear, inhomogeneous solution plus homogenous solution is another solution of the inhomogenous equation.

       For this ODE, we consider the homogenous equation: $$\frac{d^2v}{dx^2} + v = 0. $$ 
       The solution is $v(x)=A+Be^{-x}$, where A and B are constant.
       Plug the initial condition in $v(x)$, we obtain that $A+2B=0.$ 
       
       Hence, the solution can be written as $v(x)=-2B+Be^{-x},$ where $B$ is a constant.
       Thus, $u+v$ is another solution of this problem. 
    
       (b) Does a solution necessarily exist, or is there a condition that $f(x)$ must satisfy for existence? Explain.\\
      Solution:\\
      Observing the problem, let $w=u'+u$, then the problem is 
      \begin{equation*}
       \begin{cases}
           w'(x)=f(x),
           \\
           w(0)=w(l).
       \end{cases}
      \end{equation*}
      Solving this equation, 
      \begin{equation*}
       \begin{cases}
           w(x)=\int_{0}^{x}f(s)ds+C
           \\
           w(0)=w(l)=C
       \end{cases}
      \end{equation*}
Therefore, we obtain that $\int_{0}^{x}f(s)ds+C=C$, where $C$ is constant. This means that
$\int_{0}^{x}f(s)ds=0.$

Thus the necessary condition of solution existence is $\int_{0}^{x}f(s)ds=0.$

        \item Exercise 3 \smallskip \\
        Solve the boundary problem $u'' = 0$ for $0 < x < 1$ with $u'\left(0\right) + ku\left(0\right) = 0$ and $u'\left(1\right) \pm ku\left(1\right) = 0$.  Do the $+$ and $-$ cases separately.  What is special about the case $k=2$? \\
        Solution: \\
      First, we consider the $+$ case. The equation is 
      \begin{equation*}
        \begin{cases}
            u'' = 0,
            \\
            u'\left(0\right) + ku\left(0\right) = 0,
            \\
            u'\left(1\right) + ku\left(1\right) = 0.
        \end{cases}
      \end{equation*}
    The general soluton of $u(x)$ is $u(x)=ax+b$. From boundary conditions, 
    \begin{equation*}
        \begin{cases}
            u'(0)+ku(0)=a+kb=0,
            \\
            u'(1)+ku(1)=a+k(a+b)=(1+k)a+kb=0.
        \end{cases}
    \end{equation*}
    Thus 
    $$
    u(x)=\begin{cases}
        b & \text{ if } k = 0, \\
        0 & \text{ if } k \ne 0.
      \end{cases}
   $$
   Then, consider the $-$ case. The equation is 
   \begin{equation*}
    \begin{cases}
        u'' = 0,
        \\
        u'\left(0\right) + ku\left(0\right) = 0,
        \\
        u'\left(1\right) - ku\left(1\right) = 0.
    \end{cases}
  \end{equation*}
  The general soluton of $u(x)$ is still $u(x)=ax+b$. From boundary conditions, 
    \begin{equation*}
        \begin{cases}
            u'(0)+ku(0)=a+kb=0,
            \\
            u'(1)-ku(1)=a-k(a+b)=(1-k)a-kb=0.
        \end{cases}
    \end{equation*}
    Thus $$
    u(x)=\begin{cases}
        b & \text{ if } k = 0, \\
        -2bx+b & \text{ if } k = 2, \\
        0 & \text{ if } k \ne 0 \text{ and } k \ne 2. 
      \end{cases}
   $$
        
For the special case $k=2$, in $+$ case, 
\begin{equation*}
    \begin{cases}
        u(x)=0,
        \\
        u'\left(0\right) + ku\left(0\right) = a+2b = 0,
        \\
        u'\left(1\right) + ku\left(1\right) = 3a+2b = 0.
    \end{cases}
\end{equation*}
There are two boundary conditions, the solution is unique, euqal to 0.

For the special case $k=2$, in $-$ case,
\begin{equation*}
    \begin{cases}
        u(x)=-2bx+b,
        \\
        u'\left(0\right) + ku\left(0\right) = a+2b = 0,
        \\
        u'\left(1\right) + ku\left(1\right) = -a-2b = 0.
    \end{cases}
\end{equation*}
There's actually only one boundary condition, thus, the solution isn't unique, $u(x)=-2bx+b$.
    \end{enumerate}

    \pagebreak
    
    \item Strauss Section 1.6 \smallskip
    \begin{enumerate}
        \item Exercise 1 \smallskip \\
        What is the type of each of the following equations?
        \begin{enumerate}
            \item $u_{xx} - u_{xy} + 2u_y+u_{yy}-3u_{yx} + 4u = 0$. \smallskip \\
        Solution:\\
        Simplifying the equation: $u_{xx} - 4u_{xy} +u_{yy}+ 2u_y + 4u = 0.$
        
        From the general PDE form: $a_{11}u_{xx}+2a_{12}u_{xy}+a_{22}u_{yy}+a_{1}u_{x}+a_{2}u_{y}+a_{0}u=0.$
        
        Since $a_{11}=1,a_{12}=-2,a_{22}=1$, thus, $a_{12}^2=4, a_{11}a_{22}=1$, $a_{12}^2>a_{11}a_{22}$, hence this equation is Hyperbolic case.
            
            \item $9u_{xx} + 6u_{xy} + u_{yy} + u_x = 0$. \smallskip \\
            Since $a_{11}=9,a_{12}=3,a_{22}=1$, thus, $a_{12}^2=9, a_{11}a_{22}=9$, $a_{12}^2=a_{11}a_{22}$, hence this equation is Parabolic case.
        
            
        \end{enumerate}
        \item Exercise 6 \smallskip \\
        Consider the equation $3u_y + u_{xy} =0$.
        \begin{enumerate}
            \item What is its type? \smallskip \\
        Solution:\\
        Since $a_{11}=0,a_{12}=\frac{1}{2},a_{22}=0$, $a_{12}>a_{11}a_{22}$, this is Hyperbolic case.
            
            \item Find the general solution. (\emph{Hint:} Substitute $v = u_y$.) \smallskip \\
            Solution:\\
            Substitute $v = u_y$, the equation is $2v+v_{x}=0$. 
            
            By seperating variables, we obtain $\frac{dv}{v}=-3dx$, and intergrating both sides,
            we obtain:
            \[v(x,y)=e^{-3x}f(y).\]
            Then, \[u(x,y)=\int u_{y}dy= \int v(x,y) dy= F(y)e^{-3x}+g(x),\]  where $F'(y)=f(y)f(y)$ and g(x) are arbitrary functions.
            
            \item With the auxiliary conditions $u\left(x,0\right) = e^{-3x}$ and $u_y\left(x,0\right)$, does a solution exist? Is it unique? \smallskip \\
            Solution:\\
            The PDE equation is 
            \begin{equation*}
                \begin{cases}
                    u(x,y)=F(y)e^{-3x}+g(x),
                    \\
                    u(x,0)=e^{-3x},
                    \\
                    u_{y}(x,0)=0
                \end{cases}
            \end{equation*}
            Then, we plug initial conditions to $u(x,y)=F(y)e^{-3x}+g(x)$, we get
            \begin{equation*}
                \begin{cases}
                    u(x,0)=F(0)e^{-3x}+h(x)=e^{-3x},
                    \\
                    u_{y}(x,0)=f(y)e^{-3x}(x,0)=f(0)e^{-3x}.
                \end{cases}
            \end{equation*}
            Combinating these two equations, we solve that
            \begin{equation*}
                \begin{cases}
                    F(0)=constant,
                    \\
                    h(x)=e^{-3x}-F(0)e^{-3x}.
                \end{cases}
            \end{equation*}
            Therefore, the solution is 
           \[u(x,y)=F(y)e^{-3x}+e^{-3x}-F(0)e^{-3x}=[F(y)-F(0)]e^{-3x}, F(0)=0.\]
           Hence, the solution exist, as long as satisfying the former equation, and for function $F(y)$, the constrain conditions are $F(0)=0$, $f(0)=0$.
           
           Thus, the solution is not unique. 
           
           For example, $F(y)=ay^2$, $\forall a $ is a constant, there are infinite solutions $u(x,y)$ satisfying the equation.

            
        \end{enumerate}
    \end{enumerate}

    \pagebreak
    
    \item Strauss Section 2.1 \smallskip
    \begin{enumerate}
        \item Exercise 1 \smallskip \\
        Solve $u_{tt} = c^2u_{xx}$, $u\left(x,0\right) = e^x$, $u_t\left(x,0\right) = \sin{x}$. \smallskip \\
        Solution:\\
        Since $\phi(x)=u\left(x,0\right) = e^x$,$\psi(x)=u_t\left(x,0\right) = \sin{x}.$

        From (8), we know that
        \begin{align*}
            u(x,t)&=\frac{1}{2}[\phi(x+ct)+\phi(x-ct)]+\frac{1}{2c}\int_{x-ct}^{x+ct}\psi(s)ds\\
                  &=\frac{1}{2}[e^{x+ct}+e^{x-ct}]+\frac{1}{2c}\int_{x-ct}^{x+ct}sin(s)ds\\
                  &=\frac{1}{2}[e^{x+ct}+e^{x-ct}]+\frac{1}{2c}cos(x-ct)-\frac{1}{2c}cos(x+ct).
        \end{align*}
         %\[u(x,t)=\frac{1}{2}[\phi(x+ct)+\phi(x-ct)]+\frac{1}{2c}\int_{x-ct}^{x+ct}\psi(s)ds\]
        \item Exercise 8 \smallskip \\
        A \emph{sperical wave} is a solution of the three-dimensional wave equation of the form $u\left(r,t\right)$ where $r$ is the distance to the origin (The spherical coordinate). The wave equation takes the form
        $$u_{tt} = c^2\left(u_{rr} + \frac{2}{r}u_r\right) \text{    ("spherical wave equation").}$$
        \begin{enumerate}
            \item Change variables $v= ru$ to get the equation for $v$:  $v_{tt} = c^2v_{rr}$. \smallskip \\
            Solution:\\
            Let $v=ru$, then $u=\frac{v}{r}$, thus, 
            \begin{equation}\label{v-equ}
                \begin{cases}
                    v_{r}=u+ru_{r},
                    \\
                    v_{t}=ru_{t},
                    \\
                    v_{tt}=ru_{tt},
                    \\
                    v_{rr}=2u_{r}+u_{rr}.
                \end{cases}
            \end{equation}
            Plug (\ref*{v-equ}) into equation $u_{tt} = c^2\left(u_{rr} + \frac{2}{r}u_r\right)$, i.e.
            \[ \frac{v_{tt}}{r}=c^2(\frac{v_{rr}}{r}-\frac{2}{r}\cdot\frac{v_{t}}{r})+\frac{2c^2}{r}\cdot\frac{v_{t}}{r}=c^2\frac{v_{rr}}{r}.\]
            Therefore, $v_{tt}=r^2v_{rr}$.
            \item Solve for $v$ using (3) and thereby solve the spherical wave equation. \smallskip \\
            Solution:\\
            From the general solution of wave equation, 
            \[ru(x,t)=v(x,t)=f(r-ct)+g(r+ct). \]
            Thus, $u(x,t)= \frac{1}{r}f(r-ct)+\frac{1}{r}g(r+ct). $
            Differential $u(x,t)$ about $t$,
            \begin{equation*}
                \begin{cases}
                    u_{t}=\frac{c}{r}g'(r+ct)-\frac{c}{r}f'(r+ct),
                    \\
                    u_{tt}=\frac{c^2}{r}f''(r-ct)+\frac{c^2}{r}f''(r+ct),  
                \end{cases}
            \end{equation*}
            Differnetial $u(x,t)$ about $r$,
            \[ u_{r}=-\frac{1}{r^2}f(r-ct)+\frac{1}{r}f'(r-ct)-\frac{1}{r^2}g(r+ct)+\frac{1}{r}g'(r+ct).\]
        Differential $u(x,t)$ about $r$ again,
            \[u_{rr}=\frac{2}{r^3}f(r-ct)-\frac{2}{r^2}f'(r-ct)+\frac{1}{r}f''(r-ct)
        +\frac{2}{r^3}g(r+ct)-\frac{2}{r^2}g'(r+ct)+\frac{1}{r}g''(r+ct).\]
         Thus, $u_{tt} = c^2\left(u_{rr} + \frac{2}{r}u_r\right)$, which also menas $u(x,t)$ is a soluiton of spherical wave equation. 
           
            \item Use (8) to solve it with initial conditions $u\left(r,0\right) = \phi\left(r\right)$, $u_t\left(r,0\right) = \psi\left(r\right)$, taking both $\phi\left(r\right)$ and $\psi\left(r\right)$ to be even functions of $r$. \smallskip \\
            Solution:\\
            Since $v(r,t)$ is the solution of classical wave equation. Thus we will use (8) to help us solve spherical wave equation.
            From the initial condition of $u(r,t)$, we wil obtain the initial condition of $v(r,t)$
            \begin{equation*}
                \begin{cases}
                    v(r,0)=ru(r,0)=r\phi(r),
                    \\
                    v_{t}(r,0)=ru_{t}(r,0)=r\psi(r).
                \end{cases}
            \end{equation*}
            And from (8), we know the form of wave equation's solution is 
       \[v(x,t)=\frac{1}{2}[(r+ct)\phi(r+ct)+(r-ct)\phi(r-ct)]+\frac{1}{2c}\int_{r-ct}^{r+ct}s\psi(s)ds,\] therefore, the form of $u(r,t)$ is 
          \[u(r,t)=\frac{v(r,t)}{r}=\frac{1}{2}[(1+\frac{c}{r}t)\phi(r+ct)+(r-ct)\phi(1-\frac{c}{r}t)]+\frac{1}{2cr}\int_{r-ct}^{r+ct}s\psi(s)ds\]
          This is the solution of spherical wave equation.

          Then we will check if $u(x,t)$ satisfies the initial condition:
          \[u(r,0)=\frac{1}{2}[\phi(r)+\phi(r)]+\frac{1}{2cr}\int_{r}^{r}s\psi(s)ds=\phi(r).\]
          Since 

          \begin{align*}
            u_{t}(r,t)&=\frac{1}{2}[\frac{c}{r}\phi+c(1+\frac{c}{r}t)\phi'-\frac{c}{r}\phi -c(1-\frac{c}{r}t)\phi']+\frac{1}{2cr}\partial_{t}\int_{r-ct}^{r+ct}s\psi(s)ds\\
            &=\frac{1}{2cr}[c(r+ct)\psi(r+ct)+c(r-ct)\psi(r-ct)].
          \end{align*}
          Therefroe, \[u_{t}(r,0)=\frac{1}{2cr}[cr\psi(r)+cr\psi(r)]=\psi(r).\]
          Hence, $u(x,t)$ also satisfies the initial condition, in summary, $u(x,t)$ is the solution of spherical wave equation.


            
        \end{enumerate}
        
        \item Exercise 9 \smallskip \\
        Solve $u_{xx} - 3 u_{xt} - 4u_{tt} = 0$, $u\left(x,0\right) = x^2$, $u_t\left(x,0\right)$ = $e^x$. (\emph{Hint:} Factor the operator as we did for the wave equation.) \smallskip \\
        Solution:\\
        Using factor the operation to the wave equation: 
        \[u_{xx} - 3 u_{xt} - 4u_{tt}=(\partial x -4\partial t)(\partial x + \partial t) u=0.\]

        Then the second order wave equation will be transfered as two one order equations.
        \begin{equation*}
            \begin{cases}
                (\partial x + \partial t) u=v,
                \\
                (\partial x -4\partial t)v=0.
            \end{cases}
        \end{equation*}
        By using characteristic, $v(x,t)=f(x+\frac{1}{4}t)$, and $u(x,t)=f(x+\frac{1}{4}t)+g(x-t).$ 
        
        From the initial condition, 
        \begin{equation*}
            \begin{cases}
                u(x,0)=f(x)+g(x)=x^2,
                \\
                u_{t}(x,0)=\frac{1}{4}f_{t}(x)-g_{t}(x)=e^x.  
            \end{cases}
        \end{equation*}
        Intergrating $ u_{t}(x,0)=\frac{1}{4}f_{t}(x)-g_{t}(x)$ about $t$, then we get

        \[u(x,0)=\int u_{t}(x,0) dt =\int (\frac{1}{4}f_{t}(x)-g_{t}(x))dt=\frac{1}{4}f(x)-g(x)=e^x.\]
        Combinating equations:
        \begin{equation*}
            \begin{cases}
                u(x,0)=f(x)+g(x)=x^2,
                \\
                u(x,0)=\frac{1}{4}f(x)-g(x)=e^x.
            \end{cases}
        \end{equation*}
        We solve that 
        \begin{equation*}
            \begin{cases}
                f(x)=\frac{4}{5}(x^2+e^x+c),
                \\
                g(x)=\frac{1}{5}(x^2-4e^x-4c).
            \end{cases}
        \end{equation*}
        Therefore,
        \begin{align*}
            u(x,t)&=f(x+\frac{1}{4}t)+g(x-t)\\
            &=\frac{4}{5}[(x+\frac{1}{4}t)^2+e^{(x+\frac{1}{4}t)}+c] + \frac{1}{5}[(x-t)^2-4e^{(x-t)}-4c]\\
            &=\frac{4}{5}(x+\frac{1}{4}t)^2+\frac{4}{5}e^{(x+\frac{1}{4}t)}+\frac{1}{5}(x-t)^2- \frac{4}{5}e^{(x-t)}.
        \end{align*}
        Hence, $u(x,t)=\frac{4}{5}(x+\frac{1}{4}t)^2+\frac{4}{5}e^{(x+\frac{1}{4}t)}+\frac{1}{5}(x-t)^2- \frac{4}{5}e^{(x-t)}$ is the solution of this equation.
         \end{enumerate}

   \pagebreak
    
    \item Strauss Section 2.2 \smallskip
    \begin{enumerate}
        \item Exercise 1 \smallskip \\
        Use the energy conservation of the wave equation to prove that the only solution with $\phi \equiv 0$ and $\psi \equiv 0$ is $u \equiv 0$. (\emph{Hint:} Use the first vanishing theorem in Section A.1.) \smallskip \\
        Solution:\\
         First, we should show that $u\equiv0$ is a solution of the wave equation. Actually, it is trivial. 
        
        Since when $u=0$, thus $u_{tt}=0=u{xx}$, therefore, $u_{tt}=c^2u_{xx}$.
        
        Then, we should show $u=0$ is only solution with $\phi \equiv 0$ and $\psi \equiv 0$.
        
        Assume there exists another solution $v(x,t)\ne 0$ also satisfying wave equation, i.e. $v_{tt}=c^2v_{xx}.$

        Without loss of generalism, assume $v(x,t)> 0$, then by using energy equation
        \[ E(x,t)=\frac{1}{2}\int_{R} \left(v_{t}^{2}(x,t)+v_{x}^{2}(x,t)\right)dx\]

        Since the wave equation is energy conservation, i.e. $\frac{dE}{dt}=0.$ Therefore, $E = constant.$
        
        \[E(x,0)= \frac{1}{2}\int_{R}\left(v_{t}^{2}(x,0)+v_{x}^{2}(x,0)\right)dx=\frac{1}{2}\int_{R}\left(\psi^{2}+\phi_{x}^{2}\right)dx=\frac{1}{2}\int_{R}(0+0)dx =0.\]

        Thus $E(x,t)=E(x,0)=0.$

        And \[ 0=E(x,t)=\frac{1}{2}\int_{R} \left(v_{t}^{2}(x,t)+v_{x}^{2}(x,t)\right)dx,\] where $ v_{t}^{2}(x,t)+v_{x}^{2} (x,t)>0.$ 

        From Vanishing Theorem: $v_{t}^{2}(x,t)+v_{x}^{2}(x,t)=0 \implies v_{t}=0, v_{x}=0.$

        Hence, $ \nabla v(x,t)=(v_t(x,t),v_x(x,t))=(0,0),$ which means $v(x,t)=constant.$

        From initial condition, $v(x,0)= \phi =0$, thus, $v(x,t)=0, \forall x\in \mathbb{R}, t\ge 0.$

        Therefore, $u(x,t)=v(x,t)$, $u(x,t)\equiv 0$ is the only solution of wave equation with $\phi \equiv 0$ and $\psi \equiv 0$.
        
        %Let $f(x)$ be a continuous function in a finite closed interval [a,b]. Assume that $f(x)\ne 0$ in the interval and that $\int_{a}^{b}f(x)dx=0$. 
        %Then $f(x)$ is udentically zero.

        
        \item Exercise 2 \smallskip 

        For a solution $u\left(x,t\right)$ of the wave equation with $\rho = T = c = 1$, the energy density is defined as $e = \frac{1}{2}\left(u_t^2 + u_x^2\right)$ and the momentum density as $p =u_tu_x$.\smallskip 
        \begin{enumerate}
            \item Show that $\frac{\partial e}{\partial t} = \frac{\partial p}{\partial x}$ and $\frac{\partial p}{\partial t} = \frac{\partial e}{\partial x}$. \smallskip \\
            Solution: \\
            Since $e=\frac{1}{2}\left(u_{t}^{2}+u_{x}^{2}\right)$, $p=u_{t} u_{x} $.
            \begin{equation*}
                \begin{cases}
                    \frac{\partial e}{\partial t}=u_{t} u_{tt}+u_{xt} u_{x},
                    \\
                    \frac{\partial p}{\partial x}=u_{xt} u_{x}+u_{t} u_{xx}.
                \end{cases}
            \end{equation*}
            Thus, \[\frac{\partial e}{\partial t}=u_{t} u_{t t}+u_{x t} u_{x}=u_{t} u_{x x}+u_{x t} u_{x}=\frac{\partial p}{\partial x} .\]
            
            \begin{equation*}
                \begin{cases}
                    \frac{\partial p}{\partial t}=u_{tt} u_{x}+u_{t} u_{xt},
                    \\
                    \frac{\partial e}{\partial x}=u_{t} u_{tx}+u_{x} u_{xx}.
                \end{cases}
            \end{equation*}
            Hence, \[\frac{\partial p}{\partial t}=u_{tt} u_{x}+u_{t} u_{xt}=u_{t} u_{tx}+u_{x} u_{xx}=\frac{\partial e}{\partial x} .\]
            


            \item Show that both $e\left(x,t\right)$ and $p\left(x,t\right)$ also satisfy the wave equation.    
           
            Solution:\\
             First, we consider $e(x,t).$
            \[e_{tt}  =\frac{\partial}{\partial t}\left(u_{t} u_{tt}+u_{x} u_{xt}\right)=u_{tt} u_{tt}+u_{t} u_{ttt}+u_{xtt} u_{x}+u_{xt} u_{xt},\]          
            \[ e_{xx}  =\frac{\partial}{\partial x}\left(u_{t} u_{tx}+u_{x} u_{xx}\right) =u_{tx} u_{tx}+u_{t} u_{txx}+u_{xx} u_{xx}+u_{x} u_{xxx}. \]
            
            Since $u_{xx}=u_{tt}$, then 
            \[ e_{tt}=u_{tt} u_{tt}+u_{t} u_{ttt}+u_{xtt} u_{x}+u_{xt} u_{xt}=u_{tx} u_{tx}+u_{t} u_{txx}+u_{xx} u_{xx}+u_{x} u_{xxx}=e_{xx}. \]
            
            Therefore, $e(x,t)$ satisfies the wave equation.
            
 
            Then, considerinng about $p(x,t).$
            \[p_{tt}  =\frac{\partial}{\partial t}\left(u_{tt} u_{x}+u_{t} u_{xt}\right)=u_{ttt} u_{x}+u_{tt} u_{xt}+u_{tt} u_{xt}+u_{t} u_{xtt},\]          
            \[ p_{xx}  =\frac{\partial}{\partial x}\left(u_{xt} u_{x}+u_{t} u_{xx}\right) =u_{xxt} u_{x}+u_{xt} u_{xx}+u_{tx} u_{xx}+u_{t} u_{xxx}. \]
            since $u_{xx}=u_{tt}$, then 
            \[ p_{tt}=u_{ttt} u_{x}+u_{tt} u_{xt}+u_{tt} u_{xt}+u_{t} u_{xtt}=u_{xxt} u_{x}+u_{xt} u_{xx}+u_{tx} u_{xx}+u_{t} u_{xxx}=p_{xx}. \]
            
            Therefore, $p(x,t)$ satisfies the wave equation.
        \end{enumerate}
    \end{enumerate}
    
    \pagebreak

    \item Strauss Section 2.3 \smallskip
    \begin{enumerate}
        \item Exercise 1 \smallskip \\
        Consider the solution $1 - x^2 -2kt$ of the diffusion equation.  Find the locations of its maximum and its minimum in the closed rectangle $\{0 \leq x \leq 1, 0 \leq t \leq T\}$. \smallskip \\
          Solution:\\
          Let $u(x,t)=1 - x^2 -2kt$, since $u_{t}=-2k$, $u_{xx}=-2$, thus $u_{t}=ku_{xx}$, where $k>0$ is a constant. (Namely, diffusivity)
         
          From the maximum value theorem, the maximum and minimum value only appears in the boundary.
          
          And for diffusion equation, the diffusion flow will flow from high concentration to low concentration. Therefore, the maximum value appears at the initial position. 
          
          Comparing four sides, we obtain that 
          the maximum value is $u(0,0)=1$, the minimum value is $u(1,T)=-2kT$.
        
    \end{enumerate}
\end{enumerate}



\end{document}
